\documentclass[a4paper,12pt]{article}
\usepackage[utf8]{inputenc}
\usepackage[english,russian]{babel}
\usepackage[T2A]{fontenc}
\usepackage{mathtext}
\usepackage{gauss}
\usepackage{graphicx}
\usepackage{amsmath, amsfonts, amssymb}
\newtheorem{theorem}{Теорема}
\usepackage[left=2.50cm, right=2.00cm, top=2.00cm, bottom=2.00cm]{geometry} 
\usepackage{mathdots} 
\usepackage[pdftex]{lscape}
\usepackage{mathtools}
\usepackage{pgfplots}
\pgfplotsset{compat=1.9}
\usepackage{graphicx}%Вставка картинок правильная
\usepackage{tikz}
\usepackage{float}%"Плавающие" картинки
 \usepackage{relsize}
\usepackage{wrapfig}%Обтекание фигур (таблиц, картинок и прочего)
\usepackage{ tipa }
\usepackage{amsmath}
  \usepackage[unicode=true, colorlinks=true, linkcolor=blue, urlcolor=blue]{hyperref}
\linespread{1}
\newcommand{\om}{\overline{o}}
\newcommand{\OB}{\underline{O}}
\newcommand{\eps}{\varepsilon}
\newcommand{\RR}{\mathbb{R}}
\newcommand{\NN}{\mathbb{N}}
\newcommand{\CC}{\mathbb{C}}
\newcommand{\QQ}{\mathbb{Q}}
\newcommand{\ZZ}{\mathbb{Z}}
\newcommand{\dx}{\d{dx}}
\newcommand{\ph}{\varphi}
\newcommand{\F}{\mathbb{F}}
\newcommand{\E}{\mathbb{E}}    % Для кликабельных ссылок

\begin{document}
\section*{}
\begin{center}
\href{https://t.me/booarr}{Все вопросы - сюда}
\end{center}

	\tableofcontents % Вставка содержания

	\section{Теоретические вопросы}
	\subsection{Определение группы. Пример группы}
	Пусть $M$ - некоторое множество \\
	Бинарная операция на $M$ - это отображение $\circ : M\times M \to M, (a,b)\to a\circ b$\\
	Если на $M$ задана бинарная операция, то множество $(M,\circ) $ называют множесвом с бинарной операцией. \\
	$$\left(M,\circ\right) \text{ называется группой, если выполнены следующие три условаия:}$$
	$$\begin{cases}
		a\circ(b\circ c) = (a\circ b)\circ c , \forall a,b,c \in M \text{(ассоциативность)} \\
		\text{ существует нейтральный элемент } e \in M, e\circ a = a\circ e = a , \forall a\in M\\
		\forall a\in M \exists b\in M : a\circ b = b\circ a = e \\
	\end{cases}$$
	Группы матриц (с оперпцией умножение): \\
	$$GL_n(\RR) = \{A\in Mat_{n\times n}| detA\ne 0\}$$
	$$SL_n(\RR) = \{A\in Mat_{n\times n}| detA= 1\}$$
	
	\subsection{Примеры групп по сложению}
	Числовые аддитивные группы: $(\ZZ, +), (\QQ, +), (\RR, +), (\CC , +), (\ZZ_n,+)$.
		\subsection{Примеры групп по умножению}
		Числовые мультипликативные группы: $ (\QQ\backslash\{0\}, \times), (\RR\backslash\{0\}, \times), (\CC\backslash\{0\}, \times), (\ZZ_p\backslash\{0\}, \times)$, p - простое.
\subsection{Группа кватернионов}
\[
Q_8 = \left\{
\begin{bmatrix}
	1 & 0 \\
	0 & 1
\end{bmatrix},
\begin{bmatrix}
	-1 & 0 \\
	0 & -1
\end{bmatrix},
\begin{bmatrix}
	0 & 1 \\
	-1 & 0
\end{bmatrix},
\begin{bmatrix}
	0 & -1 \\
	1 & 0
\end{bmatrix},
\begin{bmatrix}
	0 & i \\
	i & 0
\end{bmatrix},
\begin{bmatrix}
	0 & -i \\
	-i & 0
\end{bmatrix},
\begin{bmatrix}
	i & 0 \\
	0 & -i
\end{bmatrix},
\begin{bmatrix}
	-i & 0 \\
	0 & i
\end{bmatrix}
\right\}.
\]
\subsection{Порядок элемента, порядок группы}
$$ \text{ Порядок элемента } g - \text{ это величина } \\$$
$$
	\text{ord}(g):=\begin{cases}
		\text{min} \{n \in \NN: g^n = e\}, \text{ если множнество непусто}\\
		\infty, \text{если множество пусто}
	\end{cases}
$$	
Порядок группы — мощность носителя группы, то есть, для конечных групп — количество элементов группы \\
\subsection{Группа подстановок. Теорема Кэли}
симметрическая группа $S_n$ - все перестановки длины n, $|S_n| = n!$\\
знакопеременная группа $A_n$ - все четные перестановки длины n $|A_n| = n!/2$\\
\textbf{ Теорема Кэли}: Всякая конечная группа $G, ordG =n$, изоморфна некторой подгруппе группы перестановок $S_n$. При этом каждый элемент группы $G $ сопоставляется с перестановкой $\pi_{a},\pi_{a}(g) = a\circ g $, где $g$ - произвольный элемент группы $G$. \href{https://ru.wikipedia.org/wiki/%D0%A2%D0%B5%D0%BE%D1%80%D0%B5%D0%BC%D0%B0_%D0%9A%D1%8D%D0%BB%D0%B8_(%D1%82%D0%B5%D0%BE%D1%80%D0%B8%D1%8F_%D0%B3%D1%80%D1%83%D0%BF%D0%BF)}{пример здесь}
\subsection{Определение циклической группы. Определение образующего элемента циклической группы}
$$\left\langle g \right\rangle  := \{g^n, n\in \ZZ \}$$ \\
Группа $G$ называется циклической, если существует такое $g\in G$, что $G = \left\langle g \right\rangle$\\
Элемент $g $  называется образающим элементом циклической группы $G$\\
\textbf{Пример}: Группы $(\ZZ, +), (\ZZ_n, + )$ при $n\geq 1$ являются цилическими. 
\subsection{Теорема Лагранжа. Следствия из теремы Лагранжа}
Множество \( aH := \{ ah \mid h \in H \} \) называется левым смежным классом элемента \( a \in G \) по подгруппе \( H \).

\textbf{Индекс подгруппы} $H$ в группе $G$ - это число левых смежных классов $G$ по $H$. Обозначение $ \left[G : H\right]$ \\
\textbf{Теорема Лагранжа}: Пусть $G$ - конечная группа, $H\subseteq G$- подгруппа; тогда $|G| = |H|\cdot \left[G : H\right]$ \\
\textbf{Следствие 1}: Пусть $G$ - конечная группа и $H\subseteq G$. Тогда $|H| $ делит $|G|$\\
\textbf{Следствие 2}: Пусть $G$ - конечная группа и $g\in G$. Тогда ord$G $ делит $|G|$\\
\textbf{Следствие 3}: Пусть $G$ - конечная группа и $|G|$ - простое число. Тогда $G$ - циклическая группа, порождаемая любым своим неединичным элементом.\\
\subsection{Определение нормальной подгруппы. Определение фактор-группы }
Подгруппа $H\subseteq G$ называется нормалььной, если $gH = Hg \forall g\in G$. Обозначение $H \triangleleft G$ \\
Пусть $H$ --- нормальная подгруппа группы $G$. Согласно определению, в этой ситуации левые и правые смежные классы $G$ по $H$ --- это одно и то же, и тогда мы будем называть их просто смежными классами.

Обозначим через $G/H$ множество всех смежных классов $G$ по $H$. Оказывается, что на $G/H$ можно ввести структуру группы.

Сначала введём на $G/H$ бинарную операцию, положив $(g_1H) \cdot (g_2H) := (g_1g_2)H$ для любых $g_1, g_2 \in G$. 

Как это понимать? Мы хотим перемножить два смежных класса и получить в результате третий смежный класс. Для этого мы берём какой-нибудь элемент $g_1$ из первого смежного класса, элемент $g_2$ из второго смежного класса и объявляем, что результатом перемножения наших двух смежных классов будет смежный класс элемента $g_1g_2$. Однако тут возникает потенциальная проблема: а вдруг при другом выборе элементов $g_1$ и $g_2$ из тех же смежных классов смежный класс элемента $g_1g_2$ окажется другим? Оказывается, в нашей ситуации такое невозможно, что доказывается так называемой проверкой корректности.

Корректность: пусть элементы $g'_1, g'_2 \in G$ таковы, что $g'_1H = g_1H$ и $g'_2H = g_2H$ (то есть $g'_1$ и $g'_2$ --- другие представители наших исходных смежных классов $g_1H$ и $g_2H$ соответственно). Тогда $g'_1 = g_1h_1$ и $g'_2 = g_2h_2$ для некоторых $h_1, h_2 \in H$. В соответствии с тем же определением должно выполняться равенство $(g'_1H) \cdot (g'_2H) = (g'_1g'_2)H$, потому нам нужно показать, что $(g'_1g'_2)H = (g_1g_2)H$. Имеем
\[
g'_1g'_2 = g_1h_1g_2h_2 = g_1g_2g_2^{-1}h_1g_2h_2 \subseteq (g_1g_2)H
\]
(в последнем переходе учтено, что $g_2^{-1}h_1g_2 \in H$ в силу нормальности подгруппы $H$), откуда вытекает $(g'_1g'_2)H = (g_1g_2)H$.

Итак, на множестве $G/H$ корректно определена бинарная операция. Теперь легко проверить, что $(G/H, \cdot)$ является группой:

ассоциативность: $((aH)(bH))(cH) = ((ab)H)(cH) = ((ab)c)H = (a(bc))H = (aH)((bc)H) = (aH)((bH)(cH))$;

нейтральный элемент --- это $eH$: $(eH)(aH) = (ea)H = aH = (ae)H = (aH)(eH)$;

обратный к $gH$ элемент --- это $g^{-1}H$: $(g^{-1}H)(gH) = (g^{-1}g)H = H = (gg^{-1})H = (gH)(g^{-1}H)$.

Как видно, все необходимые свойства вытекают из аналогичных свойств для группы $G$.\\
Группа $(G/H, \cdot ) $ называется факторгруппой группы $G$ по нормальной подгруппе $H$.
\textbf{Пример.} Пусть $G = (\mathbb{Z},+)$ и $H = n\mathbb{Z}$ для некоторого $n \in \mathbb{N}$. Тогда $G/H$ --- это знакомая нам группа вычетов $(\mathbb{Z}_n,+)$. Впрочем, некоторая тонкость тут в том, как именно определять группу $(\mathbb{Z}_n,+)$. С теоретической точки зрения наиболее удобно определение данной группы именно как факторгруппы $\mathbb{Z}/n\mathbb{Z}$. На практике же наиболее удобным для вычислений является определение <<на пальцах>>, когда рассматривается множество $\{0,1,\ldots,n-1\}$ с операцией сложения по модулю $n$. С формальной точки зрения это будет группа, отличная от $\mathbb{Z}/n\mathbb{Z}$, но изоморфная ей (про изоморфизмы см. ниже). 
\subsection{Определение гомоомрфизма групп}
Пусть $ G, F$ - две группы \\
Отображение $ \varphi: G\to F$ называется гомоморфизмом, если $\varphi(ab) = \varphi(a)\cdot\varphi(b)$ для любых $a,b \in G$. В каждой из групп своя бинарная операция (!)
\subsection{Основная теорема о гомоморфизмах}
Ядро гомоморфизма $\varphi$ - это множество Ker$\varphi:=\{g\in G|\varphi(g) = e_{F}\}\subseteq G$\\
Образ гомоморфизма $\varphi$ - это множество Im$\varphi:=\varphi(\text{G})\subseteq F$\\
\textbf{Теорема о гомоморфизме} $G/\ker\varphi \simeq \text{Im}\varphi$ (измомофно)
\subsection{Определение кольца. Примеры колец}
Кольцо - это множество R, на котором заданы две бинарные операции (+, $\cdot$), удоволетворяющие следующим условиям: \\
1)(R,+) - абелева группа( аддитивная группа кольца R) \\
2) $\forall a,b,c \in R$ $\begin{cases} a(b+c) = ab+ac \text{(левая дистрибутивность)} \\
	(a+b)c = ac + bc \text{(правая дистрибутивность)}
\end{cases}$
3) $(ab)c = a(bc) \forall a,b,c \in R$ (ассоциативность умножения)\\
4) существует элемент 1$\in R$ (называемый единицей), такой что $1\cdot a = a\cdot 1= a \forall a\in R$\\
$\ZZ, \QQ, \RR, \CC$ - числовые кольца\\
\subsection{Определение гомоморфизма колец}
Отображение $\varphi: R\to Q$ называется гомоморфизмом (колец), если $\varphi(a+b) = \varphi(a) +\varphi(b),\varphi(ab) = \varphi(a)\varphi(b) \forall a,b\in R$
\subsection{Определение идеала. Определение главного идела.}
Подмножество I кольца R называется (двусторонним) идеалом, если выполнены следующие 2 условия: \\
1) I - подгруппа по сложению \\
2) для всех $a\in I, r\in R$ выполнено $ra\in I, ar\in I$\\
Пусть R - коммутативное кольцо. С каждым элементом $a\in R$ связан идеал $(a):=\{ra| r\in R\}$\\
Идеал называется \textbf{главным}, если существует такое $a\in R$, что $I = (a)$\\
\textbf{Пример.} Для всякого $\geqslant0$ главный идела $(k)$ в кольце $\ZZ$ есть не что иное, как $k\ZZ$\\
\subsection{Построение факторкольца по идеалу $K\backslash I$}
Пусть R - произвольное кольцо, а I - идеал в R \\
Рассмотрим факторгруппу $(R/I,+)$. Её элементами являются смежные классы по идеалу $I$, то есть множества вида $a + I$, где $a \in R$. Мы хотим превратить $R/I$ в кольцо; для этого введём на $R/I$ операцию умножения, полагая $(a + I) \cdot (b + I) := ab + I$ для всех $a,b \in R$. Иными словами, чтобы перемножить два смежных класса в $R/I$, мы выбираем в каждом из них по представителю, перемножаем их и смежный класс результата объявляем произведением двух исходных смежных классов.

Как и в случае с определением факторгруппы, здесь нужна проверка корректности. Пусть $a + I = a' + I$, $b + I = b' + I$, то есть $a'$ и $b'$ --- другие представители смежных классов $a + I$ и $b + I$ соответственно. Тогда $a' = a + x$, $b' = b + y$ для некоторых $x,y \in I$. Тогда то же определение даёт $(a' + I)(b' + I) = a'b' + I$, и потому нам нужно показать, что $a'b' + I = ab + I$. В самом деле,
\[
a'b' + I = (a + x)(b + y) + I = ab + \underbrace{ay + xb + xy}_{\in I} + I = ab + I.
\]
Обратим внимание, что в последнем переходе существенно используется то, что $I$ является идеалом в $R$.\\
\subsection{Функция Эйлера. Теорема Эйлера} 
\textbf{Функция Эйлера}  $\varphi(n)$ - мультипликативная арифметическая функция, значение которой равно количеству натуральных чисел не превосходящих $n$ и взаимно простых с ним.\\
\textbf{Теорема Эйлера} - если $a(,m) = 1$, то $a^{\varphi(m)} \equiv 1 \pmod{m}, \quad \text{где } \varphi(m) \text{ — функция Эйлера.}$\\
$$\varphi(p) = p-1$$
$$\varphi(p^n) =p^n-p^{n-1}\text{,  где p - простое число}$$
\subsection{Малая теорема Ферма} 
Если p- простое число и $(a,p)=1$, то $a^{p-1} \equiv 1\pmod{p}$
\subsection{Определение поля. Пример}
Полем называется комммутативное ассоциативное кольцо с единицей, в котором $0\ne 1$ и всякий ненулевой элемент необратим.
\textbf{Примеры полей}: $\QQ, \RR,\CC$.
\subsection{Простое поле. Пример}
Простое поле - это поле, которое не имеет нетривиальных подполей
\subsection{Теорема о простом подполе}
Любое поле имеет ЕДИНСТВЕННОЕ тривиальное подполе, которое изоморфно либо полю рациональных чисел Q, либо полю целых чисел по вычету p  $(Z_{p})$.
\subsection{Характеристика поля. Пример}
\textbf{Определение 1.} \textit{Характеристикой} поля $K$ называется наименьшее натуральное число $p$, для которого $\underbrace{1 + 1 + \ldots + 1}_{p} = 0$. Если такого $p$ не существует, то говорят, что характеристика поля $K$ равна нулю.

Характеристика поля $K$ обозначается через $\operatorname{char} K$.

\textbf{Примеры.} $\operatorname{char} \mathbb{Q} = \operatorname{char} \mathbb{R} = \operatorname{char} \mathbb{C} = 0$, $\operatorname{char} \mathbb{Z}_p = p$.
\subsection{Алгебраические и трансцендентые элементы поля.}
Если $K, F$ - два поля - $K\subseteq F$, то поле F называетс расширением поля K\\
Элемент $\alpha\in F$ называется \textbf{алгебрическим} над K, если существует ненулевой многочлен $f\in K \left[x\right] $ со свойством $f(\alpha)= 0$. В противном случае элемент $\alpha$ называется \textbf{трансцендентым} над K\\
\textbf{Пример}: Рассмотрим расширение поелй $\QQ\subseteq R$. Элемент $\sqrt2$ является алгебраическим над $\QQ$, так как он аннулируется многочленом $x^2-2\in\QQ \left[x\right]$. Элементы $\pi, e$ - трансценденты над $\QQ$
\subsection{Определение простого элемента поля. Определение неприводимого многочлена над полем P}
\textbf{Простой элемент поля}- элемент, который  нельзя представить в виде произведения двух элементов, которые необратимы.
\textbf{Неприводимый многочлен над полем P} - нетривиальный многочлен, неразложимый в произвидение нетривиальных многочленом. То есть многочлен $p\in P\left[x\right]$ называется неприводимм, если не существует $q,r\in P\left[x\right]$, таких, что $p=qr$
\subsection{Построение конечного поля из $p^n$ элементов}
Пусть $h\in\ZZ_p\left[x\right]$ - неприводимый многочлен степени $n$. Тогдк мы знаем, что факторкольцо $F = \ZZ_p\left[x\right]/(h)$ является полем. Также мы занем, что F имеет размерность n как векторное пространство над $\ZZ_p$, а тогда $|F| = p^n$, то есть $F$- искоме поле из $p^n$ элементов. 
\subsection{Мультипликативаня группа конечного поля}.
Пусть $K$ - произвольное конечное поле из $p^n$ элементов (p- простое и $n\in\NN$). Имеем $\text{char} K = p$, и в частности $\ZZ_p\subseteq K$\\
Рассмотрим группу $K^{\times} := (k\backslash\{0\}, \times)$, она называется мультипликативной группой поля $K$.
\subsection{Пример конечного поля}
\textbf{Пример.} Построим поле из 4 элементов. В соответствии с описанной выше конструкцией возьмём многочлен 
\[
h = x^2 + x + 1 \in \mathbb{Z}_2[x].
\]
Поскольку \( h(0) = h(1) = 1 \neq 0 \), этот многочлен не имеет корней в \(\mathbb{Z}_2\). А так как \(\deg h = 2\), то отсюда следует, что \( h \) неприводим. Значит, факторкольцо 
\[
F = \mathbb{Z}_2[x]/(h)
\]
является искомым полем из 4 элементов. Имеем \( F = \{0, 1, \overline{x}, \overline{x} + 1\} \), где черта означает класс соответствующего элемента в факторкольце. На этом множестве операция сложения выполняется по модулю 2, а чтобы перемножить два элемента, их нужно сначала умножить как многочлены от \(\overline{x}\), а затем понизить все степени выше 1 по правилу 
\[
\overline{x}^2 = \overline{x} + 1.
\]
\section{Задачи}
\subsection{1 задание }
\subsubsection{1 вариант}
Найдите остаток от деление $2^{5432675}$ на 13
$$5432657 = 12\cdot45722 + 11  =  $$
$$2^{12k +11} \equiv  1\cdot2^{11}\equiv 1\pmod{13} $$
\subsubsection{2 вариант}
Найдите остаток от деления $48^{5n+3}$ на 11 \\
$$48 \equiv 4 \pmod{11}\longmapsto 4^{5n+3}\equiv 48^{5n+3} \pmod{11}  $$
$$4^{5n+3} \equiv  4^{5n}\cdot 4^3 \equiv 4^{5n}\cdot 9 \equiv 9 \pmod{11} $$
$$4^2 \equiv 5 \pmod{11}$$  
$$4^3 \equiv 9 \pmod{11} $$
$$4^4 \equiv 3\pmod{11} $$
$$4^5 \equiv 1 \pmod{11}$$
\subsection{2 задание }
\subsubsection{1 вариант}
Найдите обратную матрицу в поле $\ZZ_{13}$ для матрицы 
$$\begin{pmatrix}
	7 & 4 \\2 & 6\\
\end{pmatrix} = A$$
$$A\cdot A^{-1} = E$$

\[
\left(
\begin{array}{cc|cc}
7 & 4 & 1 & 0 \\
2 & 6 & 0 & 1
\end{array}
\right)\longmapsto \left(
\begin{array}{cc|cc}
7 & 4 & 1 & 0 \\
1 & 3 & 0 & 7
\end{array}
\right)\longmapsto \left( \begin{array}{cc|cc}
0 & 1 & 3 & 9 \\
1 & 3 & 0 & 7
\end{array}\right)
\longmapsto \left( \begin{array}{cc|cc}
0 & 1 & 3 & 9 \\
1 & 0 & 4 & 6
\end{array}\right) \longmapsto
\]
\[
 \left( \begin{array}{cc|cc}
	1 & 0 & 4 & 6\\
	0 & 1 & 3 & 9 
\end{array}\right) 
\]
$$A^{-1} = \begin{pmatrix}4 & 6\\ 3 & 9\end{pmatrix}$$
$$A\cdot A^{-1} = \begin{pmatrix}40 & 78 \\ 26 & 66\end{pmatrix} = E$$
\subsubsection{2 вариант}
\[ \left( \begin{array}{cc|cc}
		1 & 6 &0 & 1\\
		3 & 2 & 1 & 0 
\end{array}\right) \longmapsto  \left( \begin{array}{cc|cc}
		1 & 6 &0 & 1\\
		0 & 5 & 1 & 4 
\end{array}\right)\longmapsto  \left( \begin{array}{cc|cc}
		1 & 6 &0 & 1\\
		0 & 1 & 3 & 5 
\end{array}\right)\longmapsto   \left( \begin{array}{cc|cc}
		1 & 0 &3 & 6\\
		0 & 1 & 3 & 5 
\end{array}\right)
\]
$$A^{-1} = \begin{pmatrix}3 & 6 \\3 & 5\end{pmatrix}$$
\subsection{3 задание }
\subsubsection{1 вариант}
Найти обратный элемент к элементу 28 в поле $\ZZ_{167}$. Воспользуемся обратным алгоритмом Евклида
$$28x+167y = 1$$
$$167 = 28\cdot5 + 27$$
$$28 = 27+1$$
$$1 = 28-27 = 28- 167+28\cdot5 = 28\cdot6 -167\longmapsto 28^{-1} = 6$$
\subsubsection{2 вариант}
Найти обратный к элементу 19 в поле $\ZZ_{98}$
$$19x+98y = 1$$
$$98 = 19\cdot5 + 3$$
$$19 = 6\cdot3+1$$
$$1 = 19-6\cdot3 = 19-6\cdot\left(98-19\cdot5\right) = 19\cdot 31 -6\cdot98\longmapsto 19^{-1} = 31$$
\subsection{4 задание}
\subsubsection{1 вариант}
Найти НОД двух многочлен и его линейное выражение в поле $F_2[x]$\\
$$\left(x^5+x^2+1, x^4+x+1\right)$$
$$x^5+x^2+1 = \left(x^5+x+1\right)\cdot x + \left(x+1\right)$$
$$x^4+x+1 = \left(x^3+x^2+x\right)\cdot\left(x+1\right)+1$$
$$1 = \left(x^4+x+1\right)- \left(x^3+x^2+x\right)\cdot\left(x+1\right) = \left(x^4+x+1\right)-\left(x^3+x^2+x\right)\left(x^5+x^2+1-(x^4+x+1)\cdot x\right)$$
$$1 = \left(x^4+x+1\right)\left(x^4+x^3+x^2\right)+\left(x^5+x^2+1\right)\cdot\left(x^3+x^2+x\right)$$
\end{document}
