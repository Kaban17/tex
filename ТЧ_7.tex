\documentclass[a4paper,12pt]{article}
\usepackage[utf8]{inputenc}
\usepackage[english,russian]{babel}
\usepackage[T2A]{fontenc}
\usepackage{mathtext}
\usepackage{gauss}
\usepackage{graphicx}
\usepackage{amsmath, amsfonts, amssymb}
\newtheorem{theorem}{Теорема}
\usepackage[left=2.50cm, right=2.00cm, top=2.00cm, bottom=2.00cm]{geometry} 
\usepackage{mathdots} 
\usepackage[pdftex]{lscape}
\usepackage{mathtools}
\usepackage{pgfplots}
\pgfplotsset{compat=1.9}
\usepackage{graphicx}%Вставка картинок правильная
\usepackage{tikz}
\usepackage{float}%"Плавающие" картинки
 \usepackage{relsize}
\usepackage{wrapfig}%Обтекание фигур (таблиц, картинок и прочего)
\usepackage{ tipa }
\usepackage{amsmath}
  \usepackage[unicode=true, colorlinks=true, linkcolor=blue, urlcolor=blue]{hyperref}
\linespread{1}
\newcommand{\om}{\overline{o}}
\newcommand{\OB}{\underline{O}}
\newcommand{\eps}{\varepsilon}
\newcommand{\RR}{\mathbb{R}}
\newcommand{\NN}{\mathbb{N}}
\newcommand{\CC}{\mathbb{C}}
\newcommand{\QQ}{\mathbb{Q}}
\newcommand{\ZZ}{\mathbb{Z}}
\newcommand{\dx}{\d{dx}}
\newcommand{\ph}{\varphi}
\newcommand{\F}{\mathbb{F}}
\newcommand{\E}{\mathbb{E}}
\begin{document}
	\section*{1)}
	$$x^2 \equiv 219 \pmod{383}$$
	Чтобы это это сравнение было разрешимо, необходимо, чтобы $\left(\frac{219}{383}\right)$ был равен 1\\
	$$\left(\frac{219}{383}\right)\  = \left(\frac{3}{383}\right)\cdot \left(\frac{73}{383}\right) = 1\cdot \left(\frac{73}{383}\right), \text{ т.к } \left(383\equiv-1\pmod{12}\right)$$
	$$\left(\frac{73}{383}\right) = \left(\frac{383}{73}\right)\cdot (-1)^{6876} =  \left(\frac{18}{73}\right) = 1\cdot \left(\frac{2}{73}\right) = (-1)^{(73^2-1)/2} = 1$$
	Сравнение разрешимо. 
	\section*{2)}
	$$\left(\frac{5}{p}\right) = \left(\frac{p}{5}\right)(-1)^{\frac{p-1}{2}}$$
	
 	Рассмотрим все остатки по модулю 5 \\
 	\subsection*{a)}
 	$$p\equiv 1\pmod5$$
 	$$1\cdot (-1)^{0} = 1$$
 	 	\subsection*{b)}
 	$$p\equiv 4\equiv29 \pmod5$$
 	$$1\cdot (-1)^{18} = 1$$
  	 	\subsection*{c)}
 $$p\equiv 2\equiv7 \pmod5$$
 $$\left(\frac{2}{5}\right)\cdot (-1)^{3} = -1$$
   	 	\subsection*{d)}
 $$p\equiv -2\equiv13 \pmod5$$
 $$\left(\frac{3}{5}\right)\cdot (-1)^{6} = -1$$
 \section*{3)}
 $$x^2+2x+72\equiv\pmod{128\cdot151\cdot199}$$
 По каждому из  модулей $151, 199$(Символ Лежандра по каждому из этих чисел равен 1 с сооотвествующим знаменателем 80 и 128) сравнение имеет по 2 решения.  \\
 Теперь рассмотрим сравнениe $(x+1)^2\equiv57 \pmod{128}$ \\
 Найдем обратное по модулю $(128)$ к числу $57$ \\
 $$57^{64} \equiv 1\pmod{128}\leftrightarrow 57^{63}\equiv 9\pmod{128}$$
Теперь имеем: \\
$$9(x+1)^2\equiv1\pmod(128)\equiv (3x+3)^2$$
Это сравнение имеет 4 решения (7>3). \\
Так как каждое сравнение имеет решение, то общее количество решений равно $2\cdot2\cdot4 =16$
 \section*{5)}
$$a^{p-1}\equiv 1\pmod{p}$$
$$a^{q-1}\equiv 1\pmod{q}$$
Применим малую теорему Ферма: $2^{q-1} \equiv 1 \pmod q$:
$2^{2p} \equiv 1 \pmod q$ (используя $q = 2p + 1$)
$2^{p} \equiv 1 \pmod q$ (поделим обе стороны на $2^p$)

Теперь рассмотрим числа Мерсенна $M_p = 2^p - 1$:
$2^p \equiv 1 \pmod q$ \\
$$2^p - 1 \equiv 0 \pmod q$$

Таким образом, мы видим, что $2^p - 1$ делится на $q$. это означает, что $2^p - 1$ имеет делитель, отличный от 1 и $2^p - 1$. Следовательно, $2^p - 1$ не является простым, за исключением случая $p = 3$ (где $q = 2p + 1 = 7$). Таким образом, числа Мерсенна $M_p = 2^p - 1$ являются простыми только при $p = 3$.
\end{document}