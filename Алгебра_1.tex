\documentclass[a4paper,12pt]{article}
\usepackage[utf8]{inputenc}
\usepackage[english,russian]{babel}
\usepackage[T2A]{fontenc}
\usepackage{mathtext}
\usepackage{gauss}
\usepackage{graphicx}
\usepackage{amsmath, amsfonts, amssymb}
\newtheorem{theorem}{Теорема}
\usepackage[left=2.50cm, right=2.00cm, top=2.00cm, bottom=2.00cm]{geometry} 
\usepackage{mathdots} 
\usepackage[pdftex]{lscape}
\usepackage{mathtools}
\usepackage{pgfplots}
\pgfplotsset{compat=1.9}
\usepackage{graphicx}%Вставка картинок правильная
\usepackage{tikz}
\usepackage{float}%"Плавающие" картинки
 \usepackage{relsize}
\usepackage{wrapfig}%Обтекание фигур (таблиц, картинок и прочего)
\usepackage{ tipa }
\usepackage{amsmath}
  \usepackage[unicode=true, colorlinks=true, linkcolor=blue, urlcolor=blue]{hyperref}
\linespread{1}
\newcommand{\om}{\overline{o}}
\newcommand{\OB}{\underline{O}}
\newcommand{\eps}{\varepsilon}
\newcommand{\RR}{\mathbb{R}}
\newcommand{\NN}{\mathbb{N}}
\newcommand{\CC}{\mathbb{C}}
\newcommand{\QQ}{\mathbb{Q}}
\newcommand{\ZZ}{\mathbb{Z}}
\newcommand{\dx}{\d{dx}}
\newcommand{\ph}{\varphi}
\newcommand{\F}{\mathbb{F}}
\newcommand{\E}{\mathbb{E}}
\begin{document}
	\section*{Задание 1}
	$$m\circ n = \stackrel{\in \QQ\backslash\{0\}}{2mn}+ \stackrel{\in \QQ}\backslash\{0\}{2m}  + \stackrel{\in \QQ\backslash\{0\}}{n} +1 $$ Значит $\circ$ задает бинарную операцию на $\QQ\backslash\{0\}$\\
	Проверим выполение свойств группы $$(\QQ\backslash\{0\},\circ)$$
	\subsection*{1.Ассоциативность}
	$$a\circ(b\circ c) = a\circ(2bc+2c+2b + 1) = 2a(2bc+2b+2c+1) + 2a + 2(2bc+2b+ 2c + 1) + 1$$
	$$(a\circ b)\circ c = (2ab+ 2a + 2b+ 1)\circ c = 2c(2ab+ 2a + 2b+ 1) + 2c + 2(2ab+ 2a + 2b+ 1) + 1$$
	После раскрытия скобок и приведения подобных слагемых будет видно, что выражения совпадают.
	\subsection*{2. Нейтральный элемент}
	$$a\circ e = e\circ a = a$$
	$$a \circ e = 2 ae + 2a + 2e + 1 = a$$
	$$e = \frac{-1-a}{2(a+1)}= -\frac12$$
	\subsection*{3. Обратный элемент}
	$$a\circ b = b\circ a = -\frac12$$
	$$2ab + 2a+ 2b+ 1 = -\frac12$$
	$$b = -\frac{3+4a}{4a+4}\blacksquare$$
	\section*{Задание 2}
	$$m\circ n = \stackrel{>2}{2mn} + \stackrel{>-2}{2n}+ \stackrel{>-2}{2m} + 1> -1\in H_a \forall a \geq -1$$
	$$-\frac12>-1 \forall a,\quad e \in H_a$$
	$$$$
	\section*{Задание 3}
	Порядок элемента 1 равен 1 \\
	Порядок элемента 2 равен 18 \\
	Порядок элемента 3 равен 18 \\
	Порядок элемента 4 равен 9 \\
	Порядок элемента 5 равен 9 \\
	Порядок элемента 6 равен 9 \\
	Порядок элемента 7 равен 3 \\
	Порядок элемента 8 равен 6 \\
	Порядок элемента 9 равен 9 \\
	Порядок элемента 10 равен 18 \\
	Порядок элемента 11 равен 3 \\
	Порядок элемента 12 равен 6 \\
	Порядок элемента 13 равен 18 \\
	Порядок элемента 14 равен 18 \\
	Порядок элемента 15 равен 18 \\
	Порядок элемента 16 равен 9 \\
	Порядок элемента 17 равен 9 \\
	Порядок элемента 18 равен 2 \\
	Обратный элемент к элементу  1 равен 1 \\
	Обратный элемент к элементу  2 равен 10 \\
	Обратный элемент к элементу  3 равен 13 \\
	Обратный элемент к элементу  4 равен 5 \\
	Обратный элемент к элементу  5 равен 4 \\
	Обратный элемент к элементу  6 равен 16 \\
	Обратный элемент к элементу  7 равен 11 \\
	Обратный элемент к элементу  8 равен 12 \\
	Обратный элемент к элементу  9 равен 17 \\
	Обратный элемент к элементу  10 равен 2 \\
	Обратный элемент к элементу  11 равен 7 \\
	Обратный элемент к элементу  12 равен 8 \\
	Обратный элемент к элементу  13 равен 3 \\
	Обратный элемент к элементу  14 равен 15 \\
	Обратный элемент к элементу  15 равен 14 \\
	Обратный элемент к элементу  16 равен 6 \\
	Обратный элемент к элементу  17 равен 9 \\
	Обратный элемент к элементу  18 равен 18 \\
\section*{Задание 4}
Рассмотрим $a^n\in H , \text{где} n - \text{мнимальное из возможных }$ \\
Рассмотрим теперь $a^m \ne a^n$
$$a^m = a^{qn+r}$$
$$a^{-qn}\cdot a^{m} = a^r \in H. \text{ Если r не ноль, то мы нашли $r<n$, что противоречит минимальности после-}$$  днего. Значит, что каждый элемент  H представляется, как степень $a^m$
\end{document}
