\documentclass[a4paper,12pt]{article}
\usepackage[utf8]{inputenc}
\usepackage[english,russian]{babel}
\usepackage[T2A]{fontenc}
\usepackage{mathtext}
\usepackage{gauss}
\usepackage{graphicx}
\usepackage{amsmath, amsfonts, amssymb}
\newtheorem{theorem}{Теорема}
\usepackage[left=2.50cm, right=2.00cm, top=2.00cm, bottom=2.00cm]{geometry} 
\usepackage{mathdots} 
\usepackage[pdftex]{lscape}
\usepackage{mathtools}
\usepackage{pgfplots}
\pgfplotsset{compat=1.9}
\usepackage{graphicx}%Вставка картинок правильная
\usepackage{tikz}
\usepackage{float}%"Плавающие" картинки
 \usepackage{relsize}
\usepackage{wrapfig}%Обтекание фигур (таблиц, картинок и прочего)
\usepackage{ tipa }
\usepackage{amsmath}
  \usepackage[unicode=true, colorlinks=true, linkcolor=blue, urlcolor=blue]{hyperref}
\linespread{1}
\newcommand{\om}{\overline{o}}
\newcommand{\OB}{\underline{O}}
\newcommand{\eps}{\varepsilon}
\newcommand{\RR}{\mathbb{R}}
\newcommand{\NN}{\mathbb{N}}
\newcommand{\CC}{\mathbb{C}}
\newcommand{\QQ}{\mathbb{Q}}
\newcommand{\ZZ}{\mathbb{Z}}
\newcommand{\dx}{\d{dx}}
\newcommand{\ph}{\varphi}
\newcommand{\F}{\mathbb{F}}
\newcommand{\E}{\mathbb{E}}
\begin{document}
	\section*{1}
	Частичный порядок, в котором ровно 5 минимальных и ровно 5 максимальных элементов, может выглядеть следующим образом:
	\\
	Предположим, у нас есть множество элементов a, b, c, d, e, f, g, h, i, j. Тогда частичный порядок может быть определен следующим образом:\\
	
	Минимальные элементы: a, b, c, d, e
	Максимальные элементы: f, g, h, i, j
	Остальные элементы могут быть упорядочены произвольным образом в пределах частичного порядка.\\
	
	Например:\\
	 a < f\\
	 b < g \\
	 c < h \\
	 d < i\\
	 e < j \\
	
	Это пример частичного порядка, который удовлетворяет условию иметь ровно 5 минимальных и ровно 5 максимальных элементов.
	\section*{3}
	Так как порядок строгий, то у каждого элемнта есть непосредственно предшествующий и сам элемент является непосредственно предшесвующим для следующего элемента. Таких пар будет 26(мы выкинули первый и последний элементы). Если будет больше 26 элементов сравнимы, то потеряется строгость. Значит 50 пар точно быть не может.
	\section*{4}
	Рассмотрим пары вида $ (0, \ZZ)$ эта пара будет минимальной в порядке $\NN \times \ZZ$\\
	Теперь рассмотрим такую же пару в порядке $\ZZ \times \ZZ$. В этом порядке есть пары, меньшие $(0, \ZZ)$( например $(-1, \ZZ)$). Следовательно, порядки неизомофны.
	
	
\end{document}