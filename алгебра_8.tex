\documentclass[a4paper,12pt]{article}
\usepackage[utf8]{inputenc}
\usepackage[english,russian]{babel}
\usepackage[T2A]{fontenc}
\usepackage{mathtext}
\usepackage{gauss}
\usepackage{graphicx}
\usepackage{amsmath, amsfonts, amssymb}
\newtheorem{theorem}{Теорема}
\usepackage[left=2.50cm, right=2.00cm, top=2.00cm, bottom=2.00cm]{geometry} 
\usepackage{mathdots} 
\usepackage[pdftex]{lscape}
\usepackage{mathtools}
\usepackage{pgfplots}
\pgfplotsset{compat=1.9}
\usepackage{graphicx}%Вставка картинок правильная
\usepackage{tikz}
\usepackage{float}%"Плавающие" картинки
 \usepackage{relsize}
\usepackage{wrapfig}%Обтекание фигур (таблиц, картинок и прочего)
\usepackage{ tipa }
\usepackage{amsmath}
  \usepackage[unicode=true, colorlinks=true, linkcolor=blue, urlcolor=blue]{hyperref}
\linespread{1}
\newcommand{\om}{\overline{o}}
\newcommand{\OB}{\underline{O}}
\newcommand{\eps}{\varepsilon}
\newcommand{\RR}{\mathbb{R}}
\newcommand{\NN}{\mathbb{N}}
\newcommand{\CC}{\mathbb{C}}
\newcommand{\QQ}{\mathbb{Q}}
\newcommand{\ZZ}{\mathbb{Z}}
\newcommand{\dx}{\d{dx}}
\newcommand{\ph}{\varphi}
\newcommand{\F}{\mathbb{F}}
\newcommand{\E}{\mathbb{E}}
\begin{document}
\section*{1}
	$$\frac{9-40\sqrt[3]{6}- 6\sqrt[3]{36}}{1-3\sqrt[3]{6}- \sqrt[3]{36}} \in\QQ(x)\simeq Q[x]/(x^3-6)$$
	$$\frac{9-40\sqrt[3]{6}- 6\sqrt[3]{36}}{1-3\sqrt[3]{6}- \sqrt[3]{36}} = a_0 + a_1x+a_2x^2\to 9-40\sqrt[3]{6}- 6\sqrt[3]{36} = (1-3\sqrt[3]{6}- \sqrt[3]{36})\cdot( a_0 + a_1x+a_2x^2) $$
	Понижение степени $x^3 = 6, x^4 = 3x$\\
	$$\begin{cases}9= a_0-6a_2-18a_1\\
	-40 = a_1-a_0 -18a_2 \\
	-6 = a_2-a_2-3a_2
	\end{cases}\to \begin{cases}a_0 = 3,\\ a_1 = -1, \\a_2= 2 \end{cases}$$
		$$\frac{9-40\sqrt[3]{6}- 6\sqrt[3]{36}}{1-3\sqrt[3]{6}- \sqrt[3]{36}} = 3-x+x^2$$
		\section*{2}
		$$a = \sqrt7-\sqrt3-1\to (a-1)^2 =10 -2\sqrt{21}\to (a^2+2a-9)^2 =84\to $$
		$$x^4+4x^3-14x^2-36x-3 -\text{ минимальный многочлен для } a$$
		Теперь покажем, что $[\QQ(\sqrt7)(\sqrt3)\colon \QQ]=4$ \\
		$$[\QQ(\sqrt7)\colon Q]=  2 , min_{\text{многочлен}} = x^2-7$$ 
		$$[\QQ(\sqrt3)\colon Q(\sqrt{7})]=  2$$
		Пусть $\sqrt{3}\in \QQ(\sqrt{7})\to \sqrt{3} = a+b\sqrt7\to  7  =a^2+7b^2+2\sqrt7ab\to \begin{cases}
			7 = a^2+7b^2\\
			0 = 2\sqrt7ab
		\end{cases}\to\\ a= 0\text{ или } b=0$ - Противоречие. 
		$$[\QQ(\sqrt7)(\sqrt3)\colon \QQ]=2\cdot2 = 4, \text{ базис } \QQ(\sqrt7)(\sqrt3) =\{1,\sqrt{3},\sqrt7,\sqrt{21}\} $$
		\section*{3}
		\newcommand{\al}{\alpha}
		$x^3+x+1$ - неприводимый многочлен степени 3 над полем $\ZZ_2$. $|\ZZ_2/(x^3+x+1)| = 2^3 = 8$\\
		$$\ZZ_2/(x^3+x+1) = F_8 = \{0,1 , x, x+1, x^2, x^2+1, x^2+x, x^2+x+1, x^2+x\} =\{0,1, \al ,\al^2, \al^3, \al^4, \al^5,\al^6,\al^7,\},$$ {где }$\al$ $\text{ корень многочлена } x^3+x+1$
		$$\al^3 = \al+1, \al^4 = \al^2+\al, \al^5 = \al^2+\al+1, \al^6 = \al^2+1, \al^7 = 1$$
\begin{table}[h]
	\centering
	\begin{tabular}{c|cccccccc}
		$\times$ & 0 & 1 & $\al$ & $\al^2$ & $\al^3$ & $\al^4$ & $\al^5$ & $\al^6$ \\
		\hline
		0 & 0 & 0 & 0 & 0 & 0 & 0 & 0 & 0 \\
		1 & 0 & 1 & $\al$ & $\al^2$ & $\al^3$ & $\al^4$ & $\al^5$ & $\al^6$ \\
		$\al$ & 0 & $\al$ & $\al^2$ & $\al^3$ & $\al^4$ & $\al^5$ & $\al^6$ & 1 \\
		$\al^2$ & 0 & $\al^2$ & $\al^3$ & $\al^4$ & $\al^5$ & $\al^6$ & 1 & $\al$ \\
		$\al^3$ & 0 & $\al^3$ & $\al^4$ & $\al^5$ & $\al^6$ & 1 & $\al$ & $\al^2$ \\
		$\al^4$ & 0 & $\al^4$ & $\al^5$ & $\al^6$ & 1 & $\al$ & $\al^2$ & $\al^3$ \\
		$\al^5$ & 0 & $\al^5$ & $\al^6$ & 1 & $\al$ & $\al^2$ & $\al^3$ & $\al^4$ \\
		$\al^6$ & 0 & $\al^6$ & 1 & $\al$ & $\al^2$ & $\al^3$ & $\al^4$ & $\al^5$ \\
	\end{tabular}
	\caption{Таблица умножения в поле $F_8$}
\end{table}
Сложение идет по модулю 2. 
\section*{4}
Поле $K[\al] \subseteq K(\al)$ по определению. \\
Надо доказать включение в обратную сторону. \\
$$K[\al] - \text{  конечно, значит можно рассматривать его как векторное пространство над К} $$
Теперь рассмотрим минимальнй многочлен $p(x)$ для $\al$.\\
Он неприводим над К, его степень равна степени расширения $[K[\al]:K]$.\\
Теперь рассмотрим элемент из $K(\al)$. Любой его элемент может быть представлен в виде $\frac{g(\al)}{h(\al )}$. Но в тоже время $\frac{g(\al)}{h(\al )}\in K[\al]\to\frac{g(\al)}{h(\al )} = \gamma_0 + \gamma_1\cdot\al + \cdots + \gamma_{n}\cdot a^n $
(т.к для всех степеней, начиная с n будет формула понижения степени ).
 $\to K(\al) \subseteq K[\al] \to    K(\al)= K[\al]  $
 
\end{document}