\documentclass[a4paper,12pt]{article}
\usepackage[utf8]{inputenc}
\usepackage[english,russian]{babel}
\usepackage[T2A]{fontenc}
\usepackage{mathtext}
\usepackage{gauss}
\usepackage{graphicx}
\usepackage{amsmath, amsfonts, amssymb}
\newtheorem{theorem}{Теорема}
\usepackage[left=2.50cm, right=2.00cm, top=2.00cm, bottom=2.00cm]{geometry} 
\usepackage{mathdots} 
\usepackage[pdftex]{lscape}
\usepackage{mathtools}
\usepackage{pgfplots}
\pgfplotsset{compat=1.9}
\usepackage{graphicx}%Вставка картинок правильная
\usepackage{tikz}
\usepackage{float}%"Плавающие" картинки
 \usepackage{relsize}
\usepackage{wrapfig}%Обтекание фигур (таблиц, картинок и прочего)
\usepackage{ tipa }
\usepackage{amsmath}
  \usepackage[unicode=true, colorlinks=true, linkcolor=blue, urlcolor=blue]{hyperref}
\linespread{1}
\newcommand{\om}{\overline{o}}
\newcommand{\OB}{\underline{O}}
\newcommand{\eps}{\varepsilon}
\newcommand{\RR}{\mathbb{R}}
\newcommand{\NN}{\mathbb{N}}
\newcommand{\CC}{\mathbb{C}}
\newcommand{\QQ}{\mathbb{Q}}
\newcommand{\ZZ}{\mathbb{Z}}
\newcommand{\dx}{\d{dx}}
\newcommand{\ph}{\varphi}
\newcommand{\F}{\mathbb{F}}
\newcommand{\E}{\mathbb{E}}
\begin{document}
	\section*{}
	\subsection*{Д10.1.}
	Рассмотрим 10 вершин, котрые не связаны друг с другом. 
	Из одной вершины проведем 5 ребер, чтобы образовать вершину степени 5.
	Остались 4 вершины, степень которых нулевая. 0,0,0,0,1,1,1,1,1,5. Повтрим еще раз, чтобы получить вторую вершину степени 5.  1,1,1,1,1,1,1,1,5,5. Если еще раз провести такую манипуляцию, то получится вершина степени 2, причем не одна, а 4, но столько быть не может. Значит, такого графа не существует.
	\subsection*{Д10.2.}
	Если в графе 5 вершин, то максимальная сумма степеней вершин 20. Значит вершин не менее 5. Теперь покажем, что на 6 вершинах можно построить граф,сумма степеней вершин в котором равна 26. У 4 вершин будет степень 4, у двух остальных по 5.
	\subsection*{Д10.3.}
	Рассмотрим число $ij$, где $i \ne 0,9 \quad j\ne0,9$. Тогда каждое такое число связано с 4 числами, где i,j отличается на единцу. Получается, что такое число связано еще с 4. Так почти для каждого числа. Не так,тольк для чисел 00, 99, 10, 90. Они связаны с меньшим количеством чисел, но все равно связаны. Получается, что комопнент связности-1 
\end{document}