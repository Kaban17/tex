\documentclass[a4paper,12pt]{article}
\usepackage[utf8]{inputenc}
\usepackage[english,russian]{babel}
\usepackage[T2A]{fontenc}
\usepackage{mathtext}
\usepackage{gauss}
\usepackage{graphicx}
\usepackage{amsmath, amsfonts, amssymb}
\newtheorem{theorem}{Теорема}
\usepackage[left=2.50cm, right=2.00cm, top=2.00cm, bottom=2.00cm]{geometry} 
\usepackage{mathdots} 
\usepackage[pdftex]{lscape}
\usepackage{mathtools}
\usepackage{pgfplots}
\pgfplotsset{compat=1.9}
\usepackage{graphicx}%Вставка картинок правильная
\usepackage{tikz}
\usepackage{float}%"Плавающие" картинки
 \usepackage{relsize}
\usepackage{wrapfig}%Обтекание фигур (таблиц, картинок и прочего)
\usepackage{ tipa }
\usepackage{amsmath}
  \usepackage[unicode=true, colorlinks=true, linkcolor=blue, urlcolor=blue]{hyperref}
\linespread{1}
\newcommand{\om}{\overline{o}}
\newcommand{\OB}{\underline{O}}
\newcommand{\eps}{\varepsilon}
\newcommand{\RR}{\mathbb{R}}
\newcommand{\NN}{\mathbb{N}}
\newcommand{\CC}{\mathbb{C}}
\newcommand{\QQ}{\mathbb{Q}}
\newcommand{\ZZ}{\mathbb{Z}}
\newcommand{\dx}{\d{dx}}
\newcommand{\ph}{\varphi}
\newcommand{\F}{\mathbb{F}}
\newcommand{\E}{\mathbb{E}}
\begin{document}
	\section*{1}
	Это неверно. Вот контрпример. Треугольник со сторонами 6,7, 8. Для него выполнятеся неравенство треугольника, но произведение его сторон не делится на 60.
	\section*{2}
	\subsection*{a)}
		$$19x \equiv 2( mod 88)$$
		$$88 = 19\cdot4 + 12 $$
		$$19 = 12+7$$ 
		$$7 = 5 + 2 $$
		$$2 = 14\cdot19 - 3\cdot88$$
		$$\begin{cases}
			19x+88y = 2 \\
			19\cdot14-88\cdot3 = 2 \\
		\end{cases}\to 19(x-14) + 88(y+3) = 0$$
		$$x \equiv 14 (mod 88)$$
		\subsection*{b)}
				$$102x \equiv 9( mod 165)$$
				$$34x \equiv 3(mod 55)$$
	$$55 = 34 + 21 $$
	$$34 = 21+13$$
	$$21 = 13+ 8$$
	$$13 = 8 + 5$$
	$$3 = 5\cdot55 - 3\cdot 34$$
	$$\begin{cases}3 = 5\cdot55 - 3\cdot 34\\
	34x + 55y= 3\end{cases}\to 55(y-5) +34(x+8) = 0\to x \equiv = 47(mod 55)$$
	\section*{3}
	$$5^{p^2} \equiv1(mod p)$$
	$$5^{p-1} \equiv1(mod p)$$
	$$ 5^{p-1}\equiv5^{p^2} \equiv1(mod p)$$
	 Заметим, что 2 - решение.
	 Докажем, что других решений нет
	$$p-1\equiv  p^2 (mod p-1)\text{У этого уравнения нет решений . }$$
\section*{4}
$$a^{14} \equiv 1(mod15)$$
Чтобы найти все псевдопростые числа, нам достаточно рассмотреть a от 1 до 7. Так как степень четная, то дальше все числа повторяются. 
$$a = 1, a^{14} \equiv1 (mod15)$$
$$2^{14} = 2^6\cdot 2^6 \cdot 2^2 \equiv4 (mod15)$$
$$3^{14} = 3^6\cdot 3^6\cdot 3^2 =\equiv 6\cdot9 \equiv9(mod 15)$$
$$4^{14} = 4^2\cdot 4^2 \dots 4^2 \equiv 1 (mod15)$$
$$5^{14} = 5^2 \dots 5^2 \equiv 700\equiv \equiv mod 15$$
$$6^{14} = 6^2\dots 6^2 \equiv 6^7\equiv6^4\equiv6 (mod15)$$
$$7^{14} = 7^2 \dots 7^2 \equiv 4^7\equiv4^6\cdot 4 \equiv 4(mod 15)$$
Значит, все числа такого вида будут псевдопростыми: $15t + 1, 15n-1, 15m+4,15k-4$, где $m,n,k,t \in \NN$
\section*{5}
$$\begin{cases}
	x\equiv 2(mod 11)\\
	x\equiv1 (mod13)
\end{cases}$$
$$11x+13y = 1$$
$$13 = 11+ 2  = 2\cdot5+ 1+ 2$$
$$1 = 11\cdot6 - 13\cdot5$$
$$x =  11\cdot5 -13\cdot2\cdot5\text{ удовлетворяет обоим равенствам}$$ 
Общее решение $x\equiv 68(mod11\cdot13)$

\end{document}