\documentclass[a4paper,12pt]{article}
\usepackage[utf8]{inputenc}
\usepackage[english,russian]{babel}
\usepackage[T2A]{fontenc}
\usepackage{mathtext}
\usepackage{gauss}
\usepackage{graphicx}
\usepackage{amsmath, amsfonts, amssymb}
\newtheorem{theorem}{Теорема}
\usepackage[left=2.50cm, right=2.00cm, top=2.00cm, bottom=2.00cm]{geometry} 
\usepackage{mathdots} 
\usepackage[pdftex]{lscape}
\usepackage{mathtools}
\usepackage{pgfplots}
\pgfplotsset{compat=1.9}
\usepackage{graphicx}%Вставка картинок правильная
\usepackage{tikz}
\usepackage{float}%"Плавающие" картинки
 \usepackage{relsize}
\usepackage{wrapfig}%Обтекание фигур (таблиц, картинок и прочего)
\usepackage{ tipa }
\usepackage{amsmath}
  \usepackage[unicode=true, colorlinks=true, linkcolor=blue, urlcolor=blue]{hyperref}
\linespread{1}
\newcommand{\om}{\overline{o}}
\newcommand{\OB}{\underline{O}}
\newcommand{\eps}{\varepsilon}
\newcommand{\RR}{\mathbb{R}}
\newcommand{\NN}{\mathbb{N}}
\newcommand{\CC}{\mathbb{C}}
\newcommand{\QQ}{\mathbb{Q}}
\newcommand{\ZZ}{\mathbb{Z}}
\newcommand{\dx}{\d{dx}}
\newcommand{\ph}{\varphi}
\newcommand{\F}{\mathbb{F}}
\newcommand{\E}{\mathbb{E}}
\begin{document}
	\section*{1}
	Рассмотрим игру, начиная с позиции $(29, 30, 18)$. Запишем матрицу бутона для этой позиции. \\
	$$\begin{pmatrix}
		1 & 1 & 1 & 0 & 1 \\
		1  & 1 & 1 & 1 & 0 \\
		1 & 0 & 0 & 1 & 0
	\end{pmatrix}$$
	В первом и последних стоблцах нечетное количество единиц. Но можно убрать сколько-то камней, чтобы в каждом из столбцов было четное колиство единиц. 
		$$\begin{pmatrix}
		1 & 1 & 1 & 0 & 1 \\
		1  & 1 & 1 & 1 & 0 \\
		0 & 0 & 0 & 1 & 1
	\end{pmatrix} \text{Из послдней кучи убрали 15 камней. И эта позиция является выигршной}$$
	Значит, ответ на задачу - да $\blacksquare$
	\section*{2}
	Чтобы минимизировать результат игры, Мину необходим класть монеты номинал 10, а Максу, для максимизации результата, необходимо класть монеты номиналом 1. Если первым ходит Мин, то ходы будут такими: $10, 1 ,10, 1 \dots$ Всего таких ходов будет 32( $32\cdot11 = 352$) И последние 10 монет положит как раз Мин. Цена игры будет равна $32\cdot2 + 1 = 65$
	\section*{3}
	Первый ход делает синий. Он закрашивает одну из возможных вершин. После этого красный закрашивает одну из доступных вершин. Синий еще как-то ходит. Со второго хода красный будет закрашивать вершины симметрично ходам синего. В итоге, из-за нечетности, красный последний закрасит вершину так, что синий не сможет дальше игарть и проиграет. В итоге у красного есть выиграшная стратегия.
	\section*{4}
	Если $3\nshortmid n  $, то выиграшная стратегия есть у второго игрока. Он прсто повторяет все ходы противника до своего последнего хода. Если при этом число переменных четно, то так и последний ход. Если же нет, то второй игрок смотрит на количество единиц и если оно кратно трем, то ставит единицу, иначе ноль.  \\
	Если $3|n $ то уже первый имеет выигршную стратегию. Первой переменной он присваивает единицу, а дальше инвертирует ходы противника. В итоге это приведет к победе. 
\end{document}