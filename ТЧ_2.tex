\documentclass[a4paper,12pt]{article}
\usepackage[utf8]{inputenc}
\usepackage[english,russian]{babel}
\usepackage[T2A]{fontenc}
\usepackage{mathtext}
\usepackage{gauss}
\usepackage{graphicx}
\usepackage{amsmath, amsfonts, amssymb}
\newtheorem{theorem}{Теорема}
\usepackage[left=2.50cm, right=2.00cm, top=2.00cm, bottom=2.00cm]{geometry} 
\usepackage{mathdots} 
\usepackage[pdftex]{lscape}
\usepackage{mathtools}
\usepackage{pgfplots}
\pgfplotsset{compat=1.9}
\usepackage{graphicx}%Вставка картинок правильная
\usepackage{tikz}
\usepackage{float}%"Плавающие" картинки
 \usepackage{relsize}
\usepackage{wrapfig}%Обтекание фигур (таблиц, картинок и прочего)
\usepackage{ tipa }
\usepackage{amsmath}
  \usepackage[unicode=true, colorlinks=true, linkcolor=blue, urlcolor=blue]{hyperref}
\linespread{1}
\newcommand{\om}{\overline{o}}
\newcommand{\OB}{\underline{O}}
\newcommand{\eps}{\varepsilon}
\newcommand{\RR}{\mathbb{R}}
\newcommand{\NN}{\mathbb{N}}
\newcommand{\CC}{\mathbb{C}}
\newcommand{\QQ}{\mathbb{Q}}
\newcommand{\ZZ}{\mathbb{Z}}
\newcommand{\dx}{\d{dx}}
\newcommand{\ph}{\varphi}
\newcommand{\F}{\mathbb{F}}
\newcommand{\E}{\mathbb{E}}
\begin{document}
	\section*{1}
	\subsection*{а)}
	$$a = p_1^{\alpha_1}\cdot\dots \cdotp_s^{\alpha_s}$$
	$$b = p_1^{\beta_1}\cdot\dots\cdot p_s^{\beta_s}$$
	$$(a,b) = p_1^{min(\alpha_1,\beta_1 ) }\cdot \dots\cdotp_1^{min(\alpha_s,\beta_s ) }$$
	$$(a+b, [a,b]) = (a+b, p_1^{max(\alpha_1,\beta_1 ) }\cdot \dots\cdotp_1^{max(\alpha_s,\beta_s ) })$$
	Из суммы $ a+b $ можно повыносить минимальные степени каждого $p_i$ и выражение примет вид: $p_1^{min(\alpha_1,\beta_1 ) }\cdot \dots\cdot p_1^{min(\alpha_s,\beta_s ) }\cdot (p_1^{max(\alpha_1, \alpha_1-\beta_1)} \cdots)$ \\
	$$(p_1^{min(\alpha_1,\beta_1 ) }\cdot \dots\cdot p_1^{min(\alpha_s,\beta_s ) }\cdot (p_1^{max(\alpha_1, \alpha_1-\beta_1)} \cdots), p_1^{max(\alpha_1,\beta_1 ) }\cdot \dots\cdotp_1^{max(\alpha_s,\beta_s ) }) =p_1^{min(\alpha_1,\beta_1 ) }\cdot \dots\cdotp_1^{min(\alpha_s,\beta_s ) } $$
	\subsection*{б)}
	$$a = p_1^{\alpha_1}\cdot\dots \cdotp_s^{\alpha_s}$$
$$b = p_1^{\beta_1}\cdot\dots\cdot p_s^{\beta_s}$$
$$c = p_1^{\gamma_1}\cdot\dots\cdot p_s^{\gamma_s}$$
$$(a,b,c) = ((a,b),c) = p_1^{min(\alpha_1,\beta_1,\gamma_1 ) }\cdot \dots\cdotp_1^{min(\alpha_s,\beta_s,\gamma_s ) }$$
Без ограничение общности будем считать, что $\alpha_i \leqslant \beta_i \leqslant \gamma_i \quad \forall i$ \\
$$\frac{(a,b) (b,c)(c,a)}{(a,b,c)^2} = \frac{ p_1^{min(\alpha_1,\beta_1 ) }\cdot \dots\cdot p_1^{min(\alpha_s,\beta_s ) }\cdot p_1^{min(\gamma_1,\beta_1 ) }\cdot \dots\cdotp_1^{min(\gamma_s,\beta_s ) }\cdot  p_1^{min(\alpha_1,\gamma_1 ) }\cdot \dots\cdot p_1^{min(\alpha_s,\gamma_s ) }}{p_1^{2min(\alpha_1,\beta_1,\gamma_1 ) }\cdot \dots\cdot p_1^{2min(\alpha_s,\beta_s,\gamma_s ) }}$$ $$= p_1^{\beta_1}\dots p_s^{\beta_s}$$
$$\frac{[a,b][b,c][c,a]}{[a,b,c]} = \frac{p_1^{\beta_1}\cdots p_s^{\beta_s}p_1^{\gamma_1}\cdots p_s^{\gamma_s} p_1^{\gamma_1}\cdots p_s^{\gamma_s} }{p_1^{2\gamma_1}\cdots p_s^{2\gamma_s}}  =  p_1^{\beta_1}\dots p_s^{\beta_s} \blacksquare$$
\section*{2)}
Так как чисел, кратных 2 больше, чем чисел кратных 5, то нам надо посчитать количество чисел, кратных 5. 
Посчитаем количество чисел, кратных 5: $1000/5 = 200$ \\
Чисел, кратных 25: $1000/25 = 40$  \\
Чисел, кратных 125: $1000/125 = 8$
Чисел кратных 625 : $1$. \\
Каждая степень 5 дает свой ноль на конец числа. Следовательно нулей 249. в интервале от 1 до 1000 - 1000/5 = 200. 
- чисел, кратных 25 1000/25 
\section*{3}
$$n = p_1^{\alpha_1}\cdot\dots \cdot p_s^{\alpha_s} $$
Сумму всех делителей числа n можно предствить так: $$\sum_{b_i = 0, \cdots \alpha_i} p_1^{b_1}\cdots p_s^{b_s}$$
Раскрыв эту сумму, получим выражение 
$$(1+p_1 + \dots + p_1^{\alpha_1})\dots(1+p_s + \dots + p_s^{\alpha_s}) = \frac{p_1^{\alpha_1+1}-1}{p_1-1}\dots \frac{p_s^{\alpha_s+1}-1}{p_s-1} \blacksquare$$ 
\section*{4)}
$108 = 2^2 \cdot 3^3$\\
$\sigma(108) = \frac{2^3-1}{1}\frac{3^4-1}{2} = 2^3\cdot5\cdot7$\\
$\mathcal{T}(2^3\cdot5\cdot7 ) = 16$
\section*{5}
\subsection*{a)}
$$19x+88y = 2$$
88 = 4*19 + 12 \\
19 = 12 +  7 \\ 
12 = 7+5  \\
7 = 5 + 2 \\
2 = 7-5 = (19-12)- (12-7) = (19 - (88-4*19)) - (88-4*19-5*19) = -3*88 + 14*19  \\
частное решение $x = 14, y = -3$
$$\begin{cases}19x+88y = 2 \\
19*14-3*88 = 2\end{cases}$$
$$19(x-14)+ 88(y+3) = 0$$
$$x = 88t+14, t \in \ZZ$$
$$y = 19t-3$$
\section*{б)}
$$102x+165y = 9$$
$$34x+55y = 3$$
55  =34+21 \\
34  =21+13 \\
21 = 13+ 8\\
13 = 8 + 5\\
8  = 5+ 3 \\
3 = -8*34+5*55\\
$$x = -8, y = 5 \text{- частное решение.}$$
$$\begin{cases}
	34x+55y = 3 \\
	-8*34 + 5*55 = 3 
\end{cases}$$
$$34(x+8)+55(y-5) = 0$$
$$\begin{cases}
	x = 55t-8 \\
	y = -34t+5
	\end{cases}, t\in \ZZ$$
\end{document}