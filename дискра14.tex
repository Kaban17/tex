\documentclass[a4paper,12pt]{article}
\usepackage[utf8]{inputenc}
\usepackage[english,russian]{babel}
\usepackage[T2A]{fontenc}
\usepackage{mathtext}
\usepackage{gauss}
\usepackage{graphicx}
\usepackage{amsmath, amsfonts, amssymb}
\newtheorem{theorem}{Теорема}
\usepackage[left=2.50cm, right=2.00cm, top=2.00cm, bottom=2.00cm]{geometry} 
\usepackage{mathdots} 
\usepackage[pdftex]{lscape}
\usepackage{mathtools}
\usepackage{pgfplots}
\pgfplotsset{compat=1.9}
\usepackage{graphicx}%Вставка картинок правильная
\usepackage{tikz}
\usepackage{float}%"Плавающие" картинки
 \usepackage{relsize}
\usepackage{wrapfig}%Обтекание фигур (таблиц, картинок и прочего)
\usepackage{ tipa }
\usepackage{amsmath}
  \usepackage[unicode=true, colorlinks=true, linkcolor=blue, urlcolor=blue]{hyperref}
\linespread{1}
\newcommand{\om}{\overline{o}}
\newcommand{\OB}{\underline{O}}
\newcommand{\eps}{\varepsilon}
\newcommand{\RR}{\mathbb{R}}
\newcommand{\NN}{\mathbb{N}}
\newcommand{\CC}{\mathbb{C}}
\newcommand{\QQ}{\mathbb{Q}}
\newcommand{\ZZ}{\mathbb{Z}}
\newcommand{\dx}{\d{dx}}
\newcommand{\ph}{\varphi}
\newcommand{\F}{\mathbb{F}}
\newcommand{\E}{\mathbb{E}}
\begin{document}
	\section*{Д14.1.}
	Каждая вершина в левой доле имеет 4 соседа в правой доле. Для любого подмножества размера 4 в левой доле имеем минимум 4 соседа в правой доле. Выполняется теорема Холла и следовательно есть паросочетание размера 300. 
	\section*{Д14.2.}
	Пронумеруем вершины номерами от 1 до 2024. Так как в графе между любымитремя вершинами есть хотя бы два ребра, пусть это будут вершины 1,2 , 3. Между этими вершинами есть хотя бы 2 ребра.  Пусть между ребрами 1 2 и 2 3. Но также есть ребра между вершинами 1,3.4. Пусть между ребрами 1 4 и 3 4.  Среди этих ребер можно выделить паросочетание размера 2: 1-2, 3-4. Этот алгоритим можно продолжить и для остальных ребер и в итоге у нас будет паросочетание размера 1012.
	\section*{Д14.3.}
	Пусть есть паросочетание размера 50. Тогда у нас остается 51 вершина на построение независимого множества размера $\leqslant$ 51. Но по условию у нас есть независимое множество размера 52. Противоречие. Не существует паросочетания размера 50. 
\end{document}