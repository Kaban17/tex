\documentclass[a4paper,12pt]{article}
\usepackage[utf8]{inputenc}
\usepackage[english,russian]{babel}
\usepackage[T2A]{fontenc}
\usepackage{mathtext}
\usepackage{gauss}
\usepackage{graphicx}
\usepackage{amsmath, amsfonts, amssymb}
\newtheorem{theorem}{Теорема}
\usepackage[left=2.50cm, right=2.00cm, top=2.00cm, bottom=2.00cm]{geometry} 
\usepackage{mathdots} 
\usepackage[pdftex]{lscape}
\usepackage{mathtools}
\usepackage{pgfplots}
\pgfplotsset{compat=1.9}
\usepackage{graphicx}%Вставка картинок правильная
\usepackage{tikz}
\usepackage{float}%"Плавающие" картинки
 \usepackage{relsize}
\usepackage{wrapfig}%Обтекание фигур (таблиц, картинок и прочего)
\usepackage{ tipa }
\usepackage{amsmath}
  \usepackage[unicode=true, colorlinks=true, linkcolor=blue, urlcolor=blue]{hyperref}
\linespread{1}
\newcommand{\om}{\overline{o}}
\newcommand{\OB}{\underline{O}}
\newcommand{\eps}{\varepsilon}
\newcommand{\RR}{\mathbb{R}}
\newcommand{\NN}{\mathbb{N}}
\newcommand{\CC}{\mathbb{C}}
\newcommand{\QQ}{\mathbb{Q}}
\newcommand{\ZZ}{\mathbb{Z}}
\newcommand{\dx}{\d{dx}}
\newcommand{\ph}{\varphi}
\newcommand{\F}{\mathbb{F}}
\newcommand{\E}{\mathbb{E}}

\usepackage{amsmath}

\begin{document}
	
\section*{1}
\subsection*{a)}
$$2x^4-x^3-2x^2-2x-12  $$
$$x^5+x^4+x^2-4x-2  $$
$$f =  g\cdot(\frac12x+\frac34) + \frac74x^3+\frac72x^2+\frac72x+7 =r_1$$
$$g = r_1\cdot(\frac87x-\frac{20}{7}) + (4x^2+8)$$
$$r_1 =(4x^2+8)(\frac{7}{16}x+\frac78) $$ 
НОД = $4x^2+8$\\
$$4x^2+8 = g-(f-g(1/2x+3/4))\cdot(8/7x-20/7) = f(-8/7x+20/7)+g(1+(1/2x+3/4)(8/7x-20/7))$$
\subsection*{b)}
$$f = x^5+2x^3-x^2-4x-2, g = x^4+x^3-x^2-2x-2$$
$$f = g\cdot(5x+2) + (2x^3+x^2+2x+1)$$
$$g = r_1(2x^3+x^2+2x+1)+ (5x^2+3x+2)$$
$$r_1 = (6x+5)r_2 + 3x+5$$
$$r_2 = r_3\cdot(4x+6)$$
НОД = $3x+5$
$$r3x+5 = r_1-r_2(6x+5) = f(2x^2+5x+3)+ g(4x^3+6x^2+4x+3)$$
\section*{2}
\subsection*{a)}
$$x^5+6x^3-2x^2-12 = x^3(x^2+6)-2(x^2+6) = (x^3-2)(x^2+6) = \overset{\text{ над }\RR}{(x^2+6)(x-\sqrt[3]{2})(x^2+\sqrt[3]{2}+2^{2/3})} = $$
$$\overset{\text{ над }\CC}{(x-\sqrt[3]{2})(x-i\sqrt{6})(x+i\sqrt{6})(x-i\sqrt{6})(x-i(\sqrt[3]{2}+2^{2/3})(x+i\sqrt{6})(x-i(\sqrt[3]{2}+2^{2/3}))} $$
\subsection*{b)}
$$x^5+x^4+3x^2+x+3 = (x+1)(x^4+3x+3) = (x+1)(x+3)^2(x^2+4x+2) - \ZZ_5$$
\section*{3}
\subsection*{1)}
$$x^3+x^2+2x+1$$
$$x^3+x^2+2x+2$$
$$x^3+2x^2+2x+1$$
$$x^3+2x^2+2x+2$$
$$x^2+1$$
$$x^2+x+1$$
$$x^2+x+2$$
$$x^2+2x+1$$
$$x^2+2x+2$$
$$x+1$$
$$x+2$$
$$x$$
\subsection*{2)}
$$x^4+ax^3+bx^2+cx+d$$
$$f(0) = d\ne 0$$
$$f(1) = 1+a+b+c+d\ne0$$
$$f(2) = 1+2a+b+2c+d\ne0$$
$$\begin{cases}
	a+b+c+d \ne 2 \\
	2a + b +2c+d \ne 2
\end{cases}\to \begin{cases}
a+b+c+d \ne 2 \\
b+d\ne 2
\end{cases}$$
$$d = 1 \to \begin{cases}b = 0, a+c\ne 1, 2(a+c)\ne 1\\
b=2, a+c\ne2, 2(a+c)\ne 1
\end{cases}$$
$$d = 2 \to \begin{cases}b = 1, a+c\ne 2, 2(a+c)\ne 2\\
	b=2, a+c\ne1, 2(a+c)\ne 2
\end{cases}\to a+c=0$$
Всего $2\cdot2\cdot 3 = 12$ решений, но надо исключить 6 многочленов, которые раскладываются на приводимые квадратные многчлены. Всего 6 решений 
\section*{4}
Если есть один комлесный корень, то есть и второый комплексный корень, сопряженный первому. 
$$f = (x-ai-b)(x-ai+b)\cdot h, g = (x-ai-b)(x-ai+b)\cdot s$$
Но так как многочлены неприводимые, то больше многочлен разлагаться не может. А значит, $f, g$ пропрорциональны. 

\end{document}
