\documentclass[a4paper,12pt]{article}
\usepackage[utf8]{inputenc}
\usepackage[english,russian]{babel}
\usepackage[T2A]{fontenc}
\usepackage{mathtext}
\usepackage{gauss}
\usepackage{graphicx}
\usepackage{amsmath, amsfonts, amssymb}
\newtheorem{theorem}{Теорема}
\usepackage[left=2.50cm, right=2.00cm, top=2.00cm, bottom=2.00cm]{geometry} 
\usepackage{mathdots} 
\usepackage[pdftex]{lscape}
\usepackage{mathtools}
\usepackage{pgfplots}
\pgfplotsset{compat=1.9}
\usepackage{graphicx}%Вставка картинок правильная
\usepackage{tikz}
\usepackage{float}%"Плавающие" картинки
 \usepackage{relsize}
\usepackage{wrapfig}%Обтекание фигур (таблиц, картинок и прочего)
\usepackage{ tipa }
\usepackage{amsmath}
  \usepackage[unicode=true, colorlinks=true, linkcolor=blue, urlcolor=blue]{hyperref}
\linespread{1}
\newcommand{\om}{\overline{o}}
\newcommand{\OB}{\underline{O}}
\newcommand{\eps}{\varepsilon}
\newcommand{\RR}{\mathbb{R}}
\newcommand{\NN}{\mathbb{N}}
\newcommand{\CC}{\mathbb{C}}
\newcommand{\QQ}{\mathbb{Q}}
\newcommand{\ZZ}{\mathbb{Z}}
\newcommand{\dx}{\d{dx}}
\newcommand{\ph}{\varphi}
\newcommand{\F}{\mathbb{F}}
\newcommand{\E}{\mathbb{E}}
\begin{document}
\section*{1}
 	$$H \vartriangleleft G\iff \forall g\in G \colon gHg^{-1}\subseteq H$$
 	$$g = \begin{pmatrix}a & 0 \\b & c\end{pmatrix}, g^{-1}=  \begin{pmatrix}\frac1a & 0 \\\frac{b}{-ac} & \frac1c\end{pmatrix}$$
 	$$g\cdot\begin{pmatrix}
 		x^3 & 0 \\
 		y & x^2
 	\end{pmatrix}\cdot g^{-1}=\begin{pmatrix}
 	x^3 & 0 \\
 	bx^3\dots & x^2
 	\end{pmatrix} \in H $$
 	Значит $H \vartriangleleft G$
 	\section*{2}
 	$$G= \ZZ_{20}, H=\ZZ_{16}$$
 	$$\varphi(n) = n\cdot\varphi(1)$$
 	Гомоморфизм однозначно определяется тем, куда переходит 1.
 	$$20\cdot\varphi(1_{20}) \equiv 0\pmod {16}$$
 	$$\varphi(1_{20}) \equiv 0\pmod {4}$$
 	$$\varphi(1_{20}) \in \left\{0, 4, 8, 12\right\}$$
 	\section*{3}
 	$$H = \left\{e^{(i\pi)^{2t}}\colon t\in \QQ\right\}$$
 	$$\varphi \colon \QQ\to H$$
 	$$\varphi(q) =e^{(i\pi)^{2q}} $$
 	$$\varphi(q_1+q_2) = e^{(i\pi)^{2(q_1+q_2)}} =e^{(i\pi)^{2q_1}}\cdot e^{(i\pi)^{2q_2}} = \varphi(q_1)\cdot\varphi(q_2) \varphi\text{ - гомоморфизм.}$$
 	Очевидно, что $\varphi(q) = 1\iff q\in\ZZ$
 	$$\ker\varphi = \left\{t\in \ZZ\colon e^{(i\pi)^{2t}}\right\}=  \ZZ$$ 
 	По теореме о гомоморфизме для групп: $$Q\backslash \ZZ =Q\backslash \ker\varphi  \cong  \text{Im}\varphi = H \blacksquare$$
 	\section*{4}
 	$$1)\to 2) \iff \neg2)\to \neg1)$$
 	$$A\cap B \text{ - подгруппа в }G \text{ и следовательно в ней есть еще элементы, помимо e. Пусть этот элемент g}$$
 	$$\text{ord}g = k >1\overset{\text{ по т. Лагранжа}}{\to} k\vdots m, n \to (m,n)\ne 1$$
  В другую сторону это неверно, легко привести контрпример. Возьмем $\ZZ_5 = (0, 1, 2, 3, 4)$ и подгруппу $Z_2, Z_3 = (0,1), (0, 1, 2 )$
  $$\ZZ_2\cap\ZZ_3 = (0,1)\ne \left\{1\right\} $$
\end{document}