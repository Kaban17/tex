\documentclass[a4paper,12pt]{article}
\usepackage[utf8]{inputenc}
\usepackage[english,russian]{babel}
\usepackage[T2A]{fontenc}
\usepackage{mathtext}
\usepackage{gauss}
\usepackage{graphicx}
\usepackage{amsmath, amsfonts, amssymb}
\newtheorem{theorem}{Теорема}
\usepackage[left=2.50cm, right=2.00cm, top=2.00cm, bottom=2.00cm]{geometry} 
\usepackage{mathdots} 
\usepackage[pdftex]{lscape}
\usepackage{mathtools}
\usepackage{pgfplots}
\pgfplotsset{compat=1.9}
\usepackage{graphicx}%Вставка картинок правильная
\usepackage{tikz}
\usepackage{float}%"Плавающие" картинки
 \usepackage{relsize}
\usepackage{wrapfig}%Обтекание фигур (таблиц, картинок и прочего)
\usepackage{ tipa }
\usepackage{amsmath}
  \usepackage[unicode=true, colorlinks=true, linkcolor=blue, urlcolor=blue]{hyperref}
\linespread{1}
\newcommand{\om}{\overline{o}}
\newcommand{\OB}{\underline{O}}
\newcommand{\eps}{\varepsilon}
\newcommand{\RR}{\mathbb{R}}
\newcommand{\NN}{\mathbb{N}}
\newcommand{\CC}{\mathbb{C}}
\newcommand{\QQ}{\mathbb{Q}}
\newcommand{\ZZ}{\mathbb{Z}}
\newcommand{\dx}{\d{dx}}
\newcommand{\ph}{\varphi}
\newcommand{\F}{\mathbb{F}}
\newcommand{\E}{\mathbb{E}}
\begin{document}
\section*{1}
		Поле \(\mathbb{F}_9\) состоит из элементов \(\{0,1, 2, x,x+1, x+2, 2x, 2x+1, 2x+2\}\). \\
		 Формула понижения степени: \(x^2 = 2x+1\) .\\
		 Циклическое поле \(\F_9\) порождается элементами \((x, x+1,2x, 2x+2) \)
		 Это легко проверяется перебором. \href{https://pastebin.com/Nv3WfHt9}{Программа для перебора }. Константы точно не порождают наше поле. А остальный элементы в меньшей степени равны единице. 
		  
		
		\section*{2}
		Рассмотрим поля \( \mathbb{Z}_5[x]/(x^2 + 3) \) и \( \mathbb{Z}_5[y]/(y^2 + y + 1) \). 
		
		Пусть \(\alpha\) - корень многочлена \(x^2 + 3\) в \( \mathbb{Z}_5[x]/(x^2 + 3)\):
		\[
		\alpha^2 = -3 \equiv 2 \mod 5
		\]
		
		Пусть \(\beta\) - корень многочлена \(y^2 + y + 1\) в \( \mathbb{Z}_5[y]/(y^2 + y + 1)\):
		\[
		\beta^2 + \beta + 1 = 0
		\]
		или
		\[
		\beta^2 = -\beta - 1 \equiv 4 - \beta \mod 5
		\]
		
		Чтобы установить изоморфизм между этими полями, найдем такой элемент \(\ph(\alpha)\), который переводит \(\alpha\) в \(\beta\):
		\[
		\ph(\alpha) = a + b\beta
		\]
		где \(a, b \in \mathbb{Z}_5\).
		
		Пусть \(\ph(\alpha) = a + b\beta\). Подставим в уравнение:
		\[
		\alpha^2 = 2 \Rightarrow \ph(\alpha^2) = 2
		\]
		
		С другой стороны:
		\[
		\ph(\alpha^2) = \ph((a + b\beta)^2) = \ph(a^2 + 2ab\beta + b^2\beta^2) = a^2 + 2ab\beta + b^2\ph(\beta^2)
		\]
		
		Так как \(\beta^2 = 4 - \beta\):
		\[
		\ph(\alpha^2) = a^2 + 2ab\beta + b^2(4 - \beta) = a^2 + 4b^2 + (2ab - b^2)\beta
		\]
		
		Сравним это с \(\ph(\alpha^2) = 2\):
		\[
		a^2 + 4b^2 = 2 \mod 5
		\]
		и
		\[
		2ab - b^2 = 0 \mod 5 \Rightarrow b(2a - b) = 0
		\]
		
		Отсюда \(b \neq 0\), тогда \(2a = b\).
		
		Подставим \(a = \frac{b}{2}\) в уравнение \(a^2 + 4b^2 = 2 \mod 5\):
		\[
		\left(\frac{b}{2}\right)^2 + 4b^2 = 2 \mod 5
		\]
		
		Так как \(\frac{1}{2} \equiv 3 \mod 5\), \(a = 3b\):
		\[
		(3b)^2 + 4b^2 = 2 \mod 5 \Rightarrow 9b^2 + 4b^2 = 2 \mod 5 \Rightarrow 13b^2 = 2 \mod 5 \Rightarrow 3b^2 = 2 \mod 5 \Rightarrow b^2 = \frac{2}{3} \equiv 4 \mod 5
		\]
		
		Таким образом, \(b = 2\) (или \(b = -2 \equiv 3\)).
		
		Тогда \(a = 3b = 6 \equiv 1 \mod 5\).
		
		Таким образом, изоморфизм:
		\[
		\ph(\alpha) = 1 + 2\beta
		\]
		
		
		\[
		\ph(\alpha) = 1 + 2\beta
		\]
		Тогда это является нашим изоморфизмом.

		
		Изоморфизм между полями \( \mathbb{Z}_5[x]/(x^2 + 3) \) и \( \mathbb{Z}_5[y]/(y^2 + y + 1) \) устанавливается функцией:
		\[
		\ph(\alpha) = 1 + 2\beta
		\]
		где \(\alpha\) соответствует \(1 + 2\beta\).
		
\section*{3}
\[f(x) = x^3+x^2+1\]\\
 Все подполя поля \(\F_{2^{18}}\) имеют вид \(\F_{2^k}\), где \(k\) делитель 18. \(k\in \{1,2,3,6,9,18\}\)\\ 
 	1) \(\F_2\) -   многочлен не имеет корней.  \\
 	2)  \(\F_4\ = \{0,1,x, x+1\}\) 
	\[\begin{cases}f(0)\ne 0 \\ f(1)\ne 0 \\ f(x)\ne 0\\ f(x+1)\ne 0\end{cases}\] 	
	3) \(\F_{2^3}= \F_4\cup \{x^2, x^2+1, x^2+x, x^2+x+1\}\) 
	 \( f(x^2+1)) = 0\). Значит берем это поле. \\
	 Все оставшиеся подполя содержат  $x^2+1$. Значит они нам подходят. \\\textbf{Ответ:} \(\F_{2^3},\F_{2^6},\F_{2^9}, \F_{2^{18}} \)
	 
\end{document}