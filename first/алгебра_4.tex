\documentclass[a4paper,12pt]{article}
\usepackage[utf8]{inputenc}
\usepackage[english,russian]{babel}
\usepackage[T2A]{fontenc}
\usepackage{mathtext}
\usepackage{gauss}
\usepackage{graphicx}
\usepackage{amsmath, amsfonts, amssymb}
\newtheorem{theorem}{Теорема}
\usepackage[left=2.50cm, right=2.00cm, top=2.00cm, bottom=2.00cm]{geometry} 
\usepackage{mathdots} 
\usepackage[pdftex]{lscape}
\usepackage{mathtools}
\usepackage{pgfplots}
\pgfplotsset{compat=1.9}
\usepackage{graphicx}%Вставка картинок правильная
\usepackage{tikz}
\usepackage{float}%"Плавающие" картинки
 \usepackage{relsize}
\usepackage{wrapfig}%Обтекание фигур (таблиц, картинок и прочего)
\usepackage{ tipa }
\usepackage{amsmath}
  \usepackage[unicode=true, colorlinks=true, linkcolor=blue, urlcolor=blue]{hyperref}
\linespread{1}
\newcommand{\om}{\overline{o}}
\newcommand{\OB}{\underline{O}}
\newcommand{\eps}{\varepsilon}
\newcommand{\RR}{\mathbb{R}}
\newcommand{\NN}{\mathbb{N}}
\newcommand{\CC}{\mathbb{C}}
\newcommand{\QQ}{\mathbb{Q}}
\newcommand{\ZZ}{\mathbb{Z}}
\newcommand{\dx}{\d{dx}}
\newcommand{\ph}{\varphi}
\newcommand{\F}{\mathbb{F}}
\newcommand{\E}{\mathbb{E}}
\begin{document}
	\section*{1}
	1)$$\begin{pmatrix}
		a & b\\
		0 & c 
	\end{pmatrix}\text{ -  матрица обратима, если она не вырожденная}$$
	$$ac\ne 0\to a,c \ne 0$$
	2)Теперь рассмотрим произведение матриц $$\begin{pmatrix}
		a & b\\
		0 & c 
	\end{pmatrix}
	\begin{pmatrix}
	x & y\\
	0 & z 
	\end{pmatrix} = \begin{pmatrix}
	ax & ay+bz\\
	0 & cz 
	\end{pmatrix} = 0$$
	 Если $a,c,z = 0$, то матрица нулевая и матрица слева - левый делитель, матрица справа - правый делитель\\
	 Аналогично рассматривается случай$$\begin{pmatrix}
	x & y\\
0 & z  
	 \end{pmatrix}
	 \begin{pmatrix}
	 	a & b\\
	 	0 & c 
	 \end{pmatrix}$$ 
	 3) $$\begin{pmatrix}
	 	a & b\\
	 	0 & c 
	 \end{pmatrix}^{n} = 0 \text{ при } a,c = 0 , n\geq 2$$
	 \section*{2}
	 Пусть $(x,y-1) $ - главный идеал, т.е $(x,y-1) = (f), f \in \QQ(x,y)$\\
	 Значит, каждый элемент идеала имеет вид $f\cdot g, g \in \QQ(x,y)$\\
	 Рассмотрим многочлен из идеала   $a = f\cdot g$  в точке (x,1). 
	 Получился многочлен  от одной переменной из $\QQ(x)$. \\
	 Т.к $a(x,1)$ неприводим, то  $g =\text{const}$ или $f =\text{const}$.  
	 Иначе произведение будет зависеть не только от x, что приводит к противоречию
	 \section*{3}
	 Рассмотрим отображение $\varphi(f) = (f(\sqrt3),f(-\sqrt3))$. Докажем, что это гомоморфизм. \\
	 $$\varphi(f+g) = ((f+g)(\sqrt3),(f+g)(-\sqrt3)) =(f(\sqrt3)+g(\sqrt3),f(-\sqrt3)+g(-\sqrt3))  =$$ $$(f(\sqrt3),f(-\sqrt3)) + (g(\sqrt3),g(-\sqrt3)) =  \varphi(f) + \varphi(g)$$ 
	 $$\varphi(f\cdot g) = ((f\cdot g)(\sqrt3),(f\cdot g)(-\sqrt3)) =(f(\sqrt 3)\cdot g(\sqrt3),f(-\sqrt3)\cdot g(-\sqrt3))  =$$ $$(f(\sqrt3),f(-\sqrt3)) \cdot (g(\sqrt3),g(-\sqrt3)) =  \varphi(f) \cdot  \varphi(g)$$ 
	 $$\ker\varphi = (x^2+3x)\cdot p(x)$$
	 $$\text{Im} = \CC\otimes\CC \text{( все остальные многочлены. )}$$
	 $$\CC(x)\backslash(x^2+3x)\simeq \CC\otimes\CC$$
	 \section*{4}
	 Пусть I - идеал, удоволетворяющим всем свойствам из условия. Тогда рассмотрим элемент $r+I\in R\backslash I, r\notin I$\\
	 Рассмотрим главный идеал $rR$ .\\ Т Множество $\{ra+x\colon a\in R, x\in I\}$ будет идеалом и оно содержит I. Но т.к $\nexists J\subseteq I \to J = R $\\
	 Единица может представлена в виде $ra+x, a\in R, x\in I$
	 ТАким образом класс элемента $a$ обратен классу элемента $r$ в кольце $R\backslash I \to $ - это поле \\
	 Пусть теперь $R\backslash I  $ - поле и $\exists  J\triangleleft R, J\subseteq I$.
	 Тогда идеал $(r,I )\subseteq J,  r\in J\backslash I$. Единица не лежит в этом идеале, а значит класс элемента r необратим в фактор-кольце. Противоречие
	
	 \end{document}