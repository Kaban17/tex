\documentclass[a4paper,12pt]{article}
\usepackage[utf8]{inputenc}
\usepackage[english,russian]{babel}
\usepackage[T2A]{fontenc}
\usepackage{mathtext}
\usepackage{gauss}
\usepackage{graphicx}
\usepackage{amsmath, amsfonts, amssymb}
\newtheorem{theorem}{Теорема}
\usepackage[left=2.50cm, right=2.00cm, top=2.00cm, bottom=2.00cm]{geometry} 
\usepackage{mathdots} 
\usepackage[pdftex]{lscape}
\usepackage{mathtools}
\usepackage{pgfplots}
\pgfplotsset{compat=1.9}
\usepackage{graphicx}%Вставка картинок правильная
\usepackage{tikz}
\usepackage{float}%"Плавающие" картинки
 \usepackage{relsize}
\usepackage{wrapfig}%Обтекание фигур (таблиц, картинок и прочего)
\usepackage{ tipa }
\usepackage{amsmath}
  \usepackage[unicode=true, colorlinks=true, linkcolor=blue, urlcolor=blue]{hyperref}
\linespread{1}
\newcommand{\om}{\overline{o}}
\newcommand{\OB}{\underline{O}}
\newcommand{\eps}{\varepsilon}
\newcommand{\RR}{\mathbb{R}}
\newcommand{\NN}{\mathbb{N}}
\newcommand{\CC}{\mathbb{C}}
\newcommand{\QQ}{\mathbb{Q}}
\newcommand{\ZZ}{\mathbb{Z}}
\newcommand{\dx}{\d{dx}}
\newcommand{\ph}{\varphi}
\newcommand{\F}{\mathbb{F}}
\newcommand{\E}{\mathbb{E}}
\begin{document}
	\section*{1}
	$$\begin{pmatrix}
		-2 & 6 & 15 & 17 & -17\\
		-10 & -25 & -13 & -34 & 23 \\
		6 & -7 & -31 & -38& 33 \\
		13 & 16& -11 & 4 & 4 \\
		-21 & -36 & 3 & -24 & 12 
	\end{pmatrix}\xrightarrow{\text{УСВ}}
	\begin{pmatrix}
1 & 0 & 0 & \frac{67}{11} & -\frac{1}{11} \\
	0 & 1 & 0 & -\frac{29}{11} & -\frac{4}{11}\\
	0 & 	0 & 1 & 3 & -1 \\
	0 & 0 & 0 & 0 & 0\\
	0 & 0 & 0 & 0 & 0\\
	\end{pmatrix}$$
	$$a_{(5)} = -\frac{1}{11}a_{(1)} - \frac{4}{11}a_{(2)}-a_{(3)}$$
	$$a_{(4)} = \frac{67}{11}a_{(1)}- \frac{29}{11}a_{(2)}-3a_{(3)}$$
	$$A = \left(a_{(1)}|\quad 0 |\quad 0 |\frac{67}{11}a_{(1)} |-\frac{1}{11}a_{(1)} \right) + \left(0 |\quad a_{2}|\quad 0 |\quad -\frac{29}{11}a_{(2)}|- \frac{4}{11}a_{(2)} \right) + \left(0 |\quad 0 |\quad a_{(3)}| 3a_{(3)}| \quad -a_{(3)}\right)$$
	\section*{2}
	\subsection*{a)}
	$$A = \begin{pmatrix}
		3 & -1 & 3 \\
		-2 & 3 & 2 \\
		1 & 1 & 1
	\end{pmatrix}\xrightarrow{\text{УСВ}} \begin{pmatrix}
	1 & 0 & 0 \\
	0 & 1 & 0 \\
	0 & 0 & 1 
	\end{pmatrix}$$ 
	$$B =\begin{pmatrix}
		-3 & -5 & 3 \\
		7 & 9 & 5\\
		5 & 3 & 11 
	\end{pmatrix}\xrightarrow{\text{УСВ}} 
	\begin{pmatrix}
		1 & 0 & 0 \\
		0 & 1 & 0 \\
		0 & 0 & 1 
	\end{pmatrix}$$ 
	Эти вектора базисы в $\RR^3$
	\subsection*{б)}
	$$A\cdot C = B\text{, где } C\text{ - матрица перехода. }$$
	$$C = A^{-1}\cdot B = 
	\begin{pmatrix}
		\frac{15}{8} & \frac18 & \frac{49}{8} \\
		\frac92 & \frac72 & \frac{15}{2} \\
		-\frac{11}{8} & -\frac{5}{8} & -\frac{21}{8}
	\end{pmatrix}$$
	\subsection*{в)}
	$$\begin{pmatrix}
		-1 \\
		2 \\
		5
	\end{pmatrix} = C\cdot \begin{pmatrix}
	x_1 \\
	x_2 \\
	x_3 
	\end{pmatrix}\leftrightarrow 
	C^{-1}\cdot\begin{pmatrix}
	-1 \\
	2 \\
	5
	\end{pmatrix} = \begin{pmatrix}
	x_1 \\
	x_2 \\
	x_3 
	\end{pmatrix} = \begin{pmatrix}
	-\frac{91}{4} \\
	\frac{73}{4} \\
	\frac{75}{2}
	\end{pmatrix} $$
	\section*{3}
	Среди векторов 	$a_1-a_4$ найдем ЛНЗ. Они будут базисом в $L_1$ \\
	$$\begin{pmatrix}
		1 & 5 & -1 & -3 & \\
		-4 & -20 & 4 & -4 \\
		-2 & 5 & -3 & 2 \\
		2 & -2 & 2 & -2 \\
		-1 & 4 & -2 & -2 
	\end{pmatrix}\xrightarrow{\text{СВ}} 
	\begin{pmatrix}
		1 & 0 & \frac23 & 0 \\
		0 & 1 & -\frac13 & 0 \\
		0 & 0 & 0 & 1 \\
		0 & 0 & 0 & 0\\
		0 & 0 & 0 & 0\\
	\end{pmatrix}$$
	Вектора $a_1, a_2, a_4 $ - ЛНЗ и размер $\dim L_1 = 3$ \\
	Аналогично для $L_2$ \\
	$$\begin{pmatrix}
		3 & -7 & -9 & -3 \\
		-12 & -4 & 23 & -1 \\
		-1 & 6 & -3 & -5 \\
		2 & -6 & -8 & -4 \\
		0 & -3 & 4 & 4 
	\end{pmatrix}\xrightarrow{\text{СВ}}  
	\begin{pmatrix}
		1 & 0 & 0 & 2 \\
		0 & 1 & 0 & 0 \\
		0 & 0 & 1 & 1 \\
		0 & 0 & 0 & 0\\
		0 & 0 & 0 & 0\\
	\end{pmatrix}$$
	Вектора $b_1, b_2, b_3 $ - ЛНЗ и размер $\dim L_2 = 3$\\
	Найдем базис суммы подпространств. 
	$$\begin{pmatrix}
		1 & 5 & -3 & 3 & -7 & -9 \\
		-4 & -20 & -4 & -12 & -4 & 23 \\
		-2 & 5 & 2 & -1 & 6 & -3 \\
		2 & -2 & -2 & 2 & -6 & -8 \\
		-1 & 4 & -2 & 0 & -3 & 4 
	\end{pmatrix}\xrightarrow{\text{СВ}}  \begin{pmatrix}
	1 & 0 & 0 & 4 & -1 & 0 \\
	0 & 1 & 0 & \frac13 & 0 & 0 \\
	0 & 0 & 1 & 0 & 2 & 0 \\
	0 & 0 & 0 & 0 & 0 & 1 \\
	0 & 0 & 0 & 0 & 0 & 0
	\end{pmatrix}\to$$ $$ \text{ вектора} a_1,a_2,a_4 ,b_3 \text{ образуют базис в сумме подпространств. Его размрность 4 .}  $$

	Найдем базис и размерность пересечения. 
	$$\begin{pmatrix}
		1 & 0 & 0 & 4 & -1 & 0 \\
		0 & 1 & 0 & \frac13 & 0 & 0 \\
		0 & 0 & 1 & 0 & 2 & 0 \\
		0 & 0 & 0 & 0 & 0 & 1 \\
		0 & 0 & 0 & 0 & 0 & 0
	\end{pmatrix}$$
	Найдем ФСР 
	$$\begin{pmatrix}
		a_1 & a_2 & a_4 & b_1 & b_2 \\ 
		-4 & -\frac13 & 0 & 1 & 0 \\
		1 & 0 & -2 & 0 & 1 
	\end{pmatrix}$$
	Запишем получившиеся векторы в фундаментальную матрицу
	 $$\begin{pmatrix}
	 	-4 & 1 \\
	 	-\frac13 & 0 \\
	 	0 & -2 & \\
	 	1 & 0 \\
	 	0 &  1 
	 \end{pmatrix}$$
$$\begin{pmatrix}
	1 & 5 &  -3 & \\
	-4 & -20  & -4 \\
	-2 & 5  & 2 \\
	2 & -2  & -2 \\
	-1 & 4 & -2 
\end{pmatrix}\cdot \begin{pmatrix}
	-4 & 1 \\
-\frac13 & 0 \\
0 & -2 & \\
\end{pmatrix} = \begin{pmatrix}
-\frac{17}{3} & 7 \\
\frac{68}{3} & 4 \\
\frac{19}{3} & -6 \\
-\frac{22}{3} & 6 \\
\frac83 & 3
\end{pmatrix}$$
Первый и второй столбец матрицы - базис в $L_1 \cap L_2, \dim L_1 \cap L_2 = 2$
\end{document}