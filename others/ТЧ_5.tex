\documentclass[a4paper,12pt]{article}
\usepackage[utf8]{inputenc}
\usepackage[english,russian]{babel}
\usepackage[T2A]{fontenc}
\usepackage{mathtext}
\usepackage{gauss}
\usepackage{graphicx}
\usepackage{amsmath, amsfonts, amssymb}
\newtheorem{theorem}{Теорема}
\usepackage[left=2.50cm, right=2.00cm, top=2.00cm, bottom=2.00cm]{geometry} 
\usepackage{mathdots} 
\usepackage[pdftex]{lscape}
\usepackage{mathtools}
\usepackage{pgfplots}
\pgfplotsset{compat=1.9}
\usepackage{graphicx}%Вставка картинок правильная
\usepackage{tikz}
\usepackage{float}%"Плавающие" картинки
 \usepackage{relsize}
\usepackage{wrapfig}%Обтекание фигур (таблиц, картинок и прочего)
\usepackage{ tipa }
\usepackage{amsmath}
  \usepackage[unicode=true, colorlinks=true, linkcolor=blue, urlcolor=blue]{hyperref}
\linespread{1}
\newcommand{\om}{\overline{o}}
\newcommand{\OB}{\underline{O}}
\newcommand{\eps}{\varepsilon}
\newcommand{\RR}{\mathbb{R}}
\newcommand{\NN}{\mathbb{N}}
\newcommand{\CC}{\mathbb{C}}
\newcommand{\QQ}{\mathbb{Q}}
\newcommand{\ZZ}{\mathbb{Z}}
\newcommand{\dx}{\d{dx}}
\newcommand{\ph}{\varphi}
\newcommand{\F}{\mathbb{F}}
\newcommand{\E}{\mathbb{E}}
\begin{document}
	\section*{1}
	$$\text{По теореме Вильсона мы имеем} \quad (p-1)! \equiv -1 (mod p), $$
	$$(p-2)! c\dot -1 \equiv -1 (mod p) | \cdot(-1)$$
	$$(p-2)! \equiv 1 (mod p)$$
	\section*{2}
	Для доказательства достаточно показать, что все числа $xn+ym$ различны. \\
	Пусть нашлись такие числа $x_0\ne x1, y_0\ne y_1 $ что $x_0n + y_0 \equiv x_1n + y_1m(\mod mn)$
	$$n(x_0-x_1) + m(y_0-y_0) \equiv 0(mod mn) \leftrightarrow x_0 -x_1 + y_0 -y_1 \equiv 0 (\mod mn)$$
	Противоречие с тем, что у нас полная система вычетов. 
	\section*{3}
	$$x_0n + y_0 \equiv x_1n + y_1m\pmod {mn}$$
		$$x_0-x_1 \equiv (y_0-y_1)\pmod {mn}$$
$$x \text{ по } \pmod m \text{ имеет $\varphi(m)$ остатков.} $$
$$y \text{ по } \pmod n\text{ имеет $\varphi(n)$ остатков.} $$
$$x,y \text{ по } \pmod {mn} \text{ имеет } \varphi(m)\cdot \varphi(n) \text{ остатков}$$
Покажем, что все числа вида $xn + ym $ взимнопросты с $mn$ \\
$$(xn +ym, mn) = d \to d|mn $$
Пусть, без ограничения общности $d|m$. Тогда $d|(xn+ym)$
$\to d|x \to (x,m) = d$ Такое может быть, если $d =1$. В остальных случаях это неверно. \\
$$\text{ Очевидно, что }x,y \text{ по } \pmod {mn} \text{ имеет } \varphi(m)\cdot \varphi(n)  = \varphi(mn)$$
	\section*{4}
	$$\begin{cases}
		x \equiv 4(mod 15)\\
		x \equiv -1 (mod 16)
		x \equiv 11 (mod 17)
	\end{cases}$$
	$$x = 16t + 1 \to 16t-1 \equiv 4(mod 15)$$ \\
	$$16t \equiv 5(mod 15)$$
	$$t \equiv 5(mod 15) \to t = 15u+ 5$$
	$$x = 16(15u+5) + 1 = 15\cdot16 u + 79$$
	$$ 15\cdot16 u + 79 \equiv 11(mod 17)$$
	$$16\cdot 15 \equiv 0 (mod 17)$$
	$$u  = 17k$$
	$$x = 15\cdot16\cdot17 + 79 \leftrightarrow x\equiv 79(mod 15\cdot 16\cdot 17)$$
\end{document}