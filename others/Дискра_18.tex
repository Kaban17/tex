\documentclass[a4paper,12pt]{article}
\usepackage[utf8]{inputenc}
\usepackage[english,russian]{babel}
\usepackage[T2A]{fontenc}
\usepackage{mathtext}
\usepackage{gauss}
\usepackage{graphicx}
\usepackage{amsmath, amsfonts, amssymb}
\newtheorem{theorem}{Теорема}
\usepackage[left=2.50cm, right=2.00cm, top=2.00cm, bottom=2.00cm]{geometry} 
\usepackage{mathdots} 
\usepackage[pdftex]{lscape}
\usepackage{mathtools}
\usepackage{pgfplots}
\pgfplotsset{compat=1.9}
\usepackage{graphicx}%Вставка картинок правильная
\usepackage{tikz}
\usepackage{float}%"Плавающие" картинки
 \usepackage{relsize}
\usepackage{wrapfig}%Обтекание фигур (таблиц, картинок и прочего)
\usepackage{ tipa }
\usepackage{amsmath}
  \usepackage[unicode=true, colorlinks=true, linkcolor=blue, urlcolor=blue]{hyperref}
\linespread{1}
\newcommand{\om}{\overline{o}}
\newcommand{\OB}{\underline{O}}
\newcommand{\eps}{\varepsilon}
\newcommand{\RR}{\mathbb{R}}
\newcommand{\NN}{\mathbb{N}}
\newcommand{\CC}{\mathbb{C}}
\newcommand{\QQ}{\mathbb{Q}}
\newcommand{\ZZ}{\mathbb{Z}}
\newcommand{\dx}{\d{dx}}
\newcommand{\ph}{\varphi}
\newcommand{\F}{\mathbb{F}}
\newcommand{\E}{\mathbb{E}}

\begin{document}
	\section*{1}
	Довольно очевидно, что 6 анализов достаточно для того, чтобы узнать наличие больных в группе(просто проведем анализ каждого человека по отдельности) \\
	Для решения этой задачи будем использовать метод американских военных.  \\
	Возьмем кровь всех 6 человек и смешаем ее. Если тест дал положительный результат, то  разделяем группы на 3 человека. Если отрицательный, то отпускаем людей. \\
	Проводим тест для каждой группы отдельно. 	В зависимости от результата отпускаем людей или проводим тесты дальше. На каждом этапе у нас появляется некая вариативность ответа и мы можем построить разрешающее дерево, глубина которого как раз и будет 6. 
	$\blacksquare$ 
	\section*{2}
	Для начала сравним 1 и 2 монеты. они могут или отличаться или совпадать весом. потом сравним 1 и 4 монеты, они также могут отличаться либо совпадать весом. Если на первом взвешивании монеты совпали, а на втором нет, то 4 монета - фальшивая. \\
	Если же было наоборот, то 2 монета фальшивая.  \\
	Если же на всех этапах совпали, то 3 монета фальшивая. \\
	Значит 2 сравнений необходимо и достаточно. $\blacksquare$
	\section*{3}
	\subsection*{a)}
	Берем 2 попавшиеся монеты. Если монеты совпадают, то фальшивая монета находится в оставшейся группе.Если совпадает, то взвешиваем новые 2 монеты.Повторяем это до того, пока либо не осталось 2 монеты, либо пока не нашли фальшивую монету.Если мы дошли до конца и у нас осталось 2 монеты(или одна, если n - нечетно и в этом случае это и есть фальшивая монеты), то среди них обязательно есть фальшивая монета(так как до этого у нас не было монет). На каждом шаге мы уменьшали количество монет на 2 и поэтому достаточно $\lfloor n/2 \rfloor$ сравнений$\blacksquare$
	\subsection*{б)}
	Построим разрешающее дерево. Вершиной будет сравнение 2 до этого несравненных монет. На каждом сравнении глубина дерева будет увеличиваться на 1. и в итоге глубина дерева будет  $\lfloor n/2 \rfloor$. $\blacksquare$
	\section*{4}
	Возьмем $l  =1$ и $r = n$ \\
	Искать будем максимум будем в таких границах : $[ \lfloor \frac{l+r-1}{3}\rfloor, \lfloor \frac{r-r+1}{3}\rfloor] $ \\
	Если $a[\frac{l+r-1}{3}\rfloor] < a[ \lfloor \frac{r-r+1}{3}\rfloor]$, то 	меняем  $l  =\frac{l+r-1}{3}\rfloor $ . Иначе $r = \lfloor \frac{r-r+1}{3}\rfloor]$\\
Так будем делать до тех пор, пока l и r не станут отличаться на 2  Сложность такого алгоритма составляет $\OB(\log n)$
	\end{document}