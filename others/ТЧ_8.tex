\documentclass[a4paper,12pt]{article}
\usepackage[utf8]{inputenc}
\usepackage[english,russian]{babel}
\usepackage[T2A]{fontenc}
\usepackage{mathtext}
\usepackage{gauss}
\usepackage{graphicx}
\usepackage{amsmath, amsfonts, amssymb}
\newtheorem{theorem}{Теорема}
\usepackage[left=2.50cm, right=2.00cm, top=2.00cm, bottom=2.00cm]{geometry} 
\usepackage{mathdots} 
\usepackage[pdftex]{lscape}
\usepackage{mathtools}
\usepackage{pgfplots}
\pgfplotsset{compat=1.9}
\usepackage{graphicx}%Вставка картинок правильная
\usepackage{tikz}
\usepackage{float}%"Плавающие" картинки
 \usepackage{relsize}
\usepackage{wrapfig}%Обтекание фигур (таблиц, картинок и прочего)
\usepackage{ tipa }
\usepackage{amsmath}
  \usepackage[unicode=true, colorlinks=true, linkcolor=blue, urlcolor=blue]{hyperref}
\linespread{1}
\newcommand{\om}{\overline{o}}
\newcommand{\OB}{\underline{O}}
\newcommand{\eps}{\varepsilon}
\newcommand{\RR}{\mathbb{R}}
\newcommand{\NN}{\mathbb{N}}
\newcommand{\CC}{\mathbb{C}}
\newcommand{\QQ}{\mathbb{Q}}
\newcommand{\ZZ}{\mathbb{Z}}
\newcommand{\dx}{\d{dx}}
\newcommand{\ph}{\varphi}
\newcommand{\F}{\mathbb{F}}
\newcommand{\E}{\mathbb{E}}
\begin{document}
	\section*{1}
	$$x^2\equiv 2\pmod{143}$$
	Чтобы это сравнение было разрешимо, необходимо и достаточно, чтобы  символ Якоби $\left(\frac{	2}{143}\right) = 1$ \\
	$$\left(\frac{2}{143}\right) =\left(\frac{2}{13}\right)\cdot \left(\frac{2}{11}\right) = (-1)^{168/8}\cdot (-1)^{120/8} = 1   $$
	\textbf{Ответ: } Сравнение разрешимо.
	\section*{2}
	Мы знаем, что: 
	$$\sum\limits_{n=1}^{1000}\left(\frac{2n-1}{1001}\right) = 0$$
	И еще мы знаем, что ровно половина чисел якоби равна $-1$. И поэтому $$\sum\limits_{n=1}^{500}\left(\frac{2n-1}{1001}\right) = 0$$
	\textbf{Ответ :} $\sum\limits_{n=1}^{500}\left(\frac{2n-1}{1001}\right) = 0$
	\section*{3}
	Числа $2, 3, 6, 7,8 $ -ПК по модулю 11 \\
	Числа $4, 5, 9, 10$ \\
	$$4^5\equiv1 \pmod{11}$$
	$$5^5 \equiv1 \pmod{11}$$
	$$9^5\equiv 1\pmod{11}$$
	$$10^2\equiv1\pmod{11}$$
		\textbf{Ответ: } $2, 3, 6, 7,8 $ 
	\section*{4}
 $g$ - ПК по модую m. Пусть $g\equiv x^2\pmod{m}$\\
 Тогда рассмотрим  $g^{\frac{\varphi(m)}{2}}$
 $$g^{\frac{\varphi(m)}{2}} \equiv (x^2)^{\varphi(m)/2} \equiv x^{\varphi(m)} \equiv 1\pmod{m}\text{Теорема Эйлера}$$
 Противоречие, так как $g $ - первообразный корень $\blacksquare$
 \section*{5}
  Пусть $g$ - ПК $\pmod{p}$, где $p $ простое
 	$$(p-1)! = \overbrace{1\cdot 2\cdot \dots \cdot(p-1)}^{\text{полная система вычетво}} \equiv \overbrace{g^1\cdot g^2\cdot \dots \cdot g^{(p-1)}}^{\text{тоже полная система вычетво}}\equiv g^{\frac{(p-1)(p-2)}{2}\cdot }\equiv 1^{(p-2)/2} \equiv1 \mod{p}\quad\blacksquare$$ 
\end{document}		