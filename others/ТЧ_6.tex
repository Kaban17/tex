\documentclass[a4paper,12pt]{article}
\usepackage[utf8]{inputenc}
\usepackage[english,russian]{babel}
\usepackage[T2A]{fontenc}
\usepackage{mathtext}
\usepackage{gauss}
\usepackage{graphicx}
\usepackage{amsmath, amsfonts, amssymb}
\newtheorem{theorem}{Теорема}
\usepackage[left=2.50cm, right=2.00cm, top=2.00cm, bottom=2.00cm]{geometry} 
\usepackage{mathdots} 
\usepackage[pdftex]{lscape}
\usepackage{mathtools}
\usepackage{pgfplots}
\pgfplotsset{compat=1.9}
\usepackage{graphicx}%Вставка картинок правильная
\usepackage{tikz}
\usepackage{float}%"Плавающие" картинки
 \usepackage{relsize}
\usepackage{wrapfig}%Обтекание фигур (таблиц, картинок и прочего)
\usepackage{ tipa }
\usepackage{amsmath}
  \usepackage[unicode=true, colorlinks=true, linkcolor=blue, urlcolor=blue]{hyperref}
\linespread{1}
\newcommand{\om}{\overline{o}}
\newcommand{\OB}{\underline{O}}
\newcommand{\eps}{\varepsilon}
\newcommand{\RR}{\mathbb{R}}
\newcommand{\NN}{\mathbb{N}}
\newcommand{\CC}{\mathbb{C}}
\newcommand{\QQ}{\mathbb{Q}}
\newcommand{\ZZ}{\mathbb{Z}}
\newcommand{\dx}{\d{dx}}
\newcommand{\ph}{\varphi}
\newcommand{\F}{\mathbb{F}}
\newcommand{\E}{\mathbb{E}}
\begin{document}
	\section*{1}
	$$x^{242}-1 \equiv 0 \pmod {286}\leftrightarrow \begin{cases}x^{242}-1 \equiv 0 \pmod {2} \\
	x^{242}-1 \equiv 0 \pmod {11}\\
	x^{242}-1 \equiv 0 \pmod {13}\end{cases}\leftrightarrow \begin{cases}x \equiv 1 \pmod {2} \\
	x\equiv \pm1 \pmod {11}\\
	x \equiv \pm1 \pmod {13}\end{cases} \leftrightarrow $$
	$$\begin{cases}
		x \equiv 1\pmod{286}(+1_{11}, +1_{13})\\
		x\equiv 131 \pmod{286} (-1_{11}, +1_{13})\\
		x\equiv 155 \pmod{286}(+1_{11}, -1_{13})\\
		x\equiv 285 \pmod{286}(-1_{11}, -1_{13})
	\end{cases}$$
	\section*{2}
	$$x^2-1\equiv0 \pmod{2^{10}\cdot3^{11}\cdot 5^{12}\cdot7^{13}}\leftrightarrow \begin{cases}x^2-1\equiv0 \pmod{{2^{10}\cdot3^{11}}}\\
		x^2-1\equiv0 \pmod{{5^{12}\cdot7^{13}}}
	\end{cases}\leftrightarrow$$
	$$\begin{cases}(x-1)(x+1)\equiv0 \pmod{{2^{10}\cdot3^{11}}}\\
		(x-1)(x+1)\equiv0\pmod{{5^{12}\cdot7^{13}}}
	\end{cases} \leftrightarrow 
	\begin{cases}
		(x-1)(x+1)\equiv0 \pmod{{2^{10}}}\\
		(x-1)(x+1)\equiv0 \pmod{{3^{11}}}\\
		(x-1)(x+1)\equiv0\pmod{{5^{12}}}\\
			(x-1)(x+1)\equiv0\pmod{{7^{13}}}\
	\end{cases}$$ Как видно, эта система имеет 10 решений.
	4 решения от первого уравнения(\textpm 1, \textpm 511), и по 2 решения от остальных уравнения(\textpm 1).
	\section*{3}
	$$x^2\equiv y^2 \pmod p \leftrightarrow (x-y)(x+y)\equiv 0 \pmod p$$
	Для каждого $x $ у нас есть пара $(+y,-y)$, которая дает ответ. Так как $x $ пробегает полную систему вычетов, то всего $2p$, но ноль и минус ноль это одно и тоже, так что ответ $2p-1$
	\section*{4}
	$(x^2-ab)(x^2-bc)(x^2-ac)\equiv0 \pmod p$  \\
	Рассмотрим символ Лежандра для каждой пары. Без ограничения общности рассмотрим только пару $ab$ \\
	$$1) \left(\frac{ab}{p}\right) = 0 $$ То есть $p|ab$ и $x = 0$ - решение.\\
	$$2) \left(\frac{ab}{p}\right) = 1$$ 
	$ab - $ квадратичный вычет и есть решение, равное $\sqrt{ab} \in \ZZ$.  \\
	$$3) \left(\frac{ab}{p}\right) = -1$$
	В таком случае смотрим на пару, где символ Лежандра равен ноль или один. 
	\section*{5}
	\subsection*{a)}
	$$\sum_{x = 0}^{58} \left(\frac{15x+79}{59}\right) = 0, $$
	$59- $ простое число. 
	\subsection*{б)}
	$$\sum_{x = 0}^{57} \left(\frac{15x+79}{59}\right) = 0-1 = -1 \left(\frac{15\cdot58+79}{59}\right) $$
	$$15\cdot58+79\equiv934\equiv49\pmod{59}$$
	$$\left(\frac{49}{59}\right) =1 $$
\end{document}