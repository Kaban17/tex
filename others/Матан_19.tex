\documentclass[a4paper,12pt]{article}
\usepackage[utf8]{inputenc}
\usepackage[english,russian]{babel}
\usepackage[T2A]{fontenc}
\usepackage{mathtext}
\usepackage{gauss}
\usepackage{graphicx}
\usepackage{amsmath, amsfonts, amssymb}
\newtheorem{theorem}{Теорема}
\usepackage[left=2.50cm, right=2.00cm, top=2.00cm, bottom=2.00cm]{geometry} 
\usepackage{mathdots} 
\usepackage[pdftex]{lscape}
\usepackage{mathtools}
\usepackage{pgfplots}
\pgfplotsset{compat=1.9}
\usepackage{graphicx}%Вставка картинок правильная
\usepackage{tikz}
\usepackage{float}%"Плавающие" картинки
 \usepackage{relsize}
\usepackage{wrapfig}%Обтекание фигур (таблиц, картинок и прочего)
\usepackage{ tipa }
\usepackage{amsmath}
  \usepackage[unicode=true, colorlinks=true, linkcolor=blue, urlcolor=blue]{hyperref}
\linespread{1}
\newcommand{\om}{\overline{o}}
\newcommand{\OB}{\underline{O}}
\newcommand{\eps}{\varepsilon}
\newcommand{\RR}{\mathbb{R}}
\newcommand{\NN}{\mathbb{N}}
\newcommand{\CC}{\mathbb{C}}
\newcommand{\QQ}{\mathbb{Q}}
\newcommand{\ZZ}{\mathbb{Z}}
\newcommand{\dx}{\d{dx}}
\newcommand{\ph}{\varphi}
\newcommand{\F}{\mathbb{F}}
\newcommand{\E}{\mathbb{E}}
\begin{document}
	\section*{1)}
	\subsection*{a)}
	$$\int\limits_0^{+\infty}\frac{\cos{7x}}{\sqrt[3]{x^2+1}}\dx =\lim_{x\to+\infty}\frac{\sin{7x}}{7\sqrt[3]{x^2+1}}+\frac{2}{21}\int\limits_0^{+\infty}\frac{\cos{7x}\cdot x}{(x^2+1)\sqrt[3]{x^2+1}} \sim\int\limits_1^{+\infty}\frac{\cos{7x}\cdot x}{(x^2+1)\sqrt[3]{x^2+1}} $$
	Возьмем $g(x) = \frac{1}{x^{5/3}}$. Эта функция $g(x)\ge f(x) = \frac{\cos{7x}\cdot x}{(x^2+1)\sqrt[3]{x^2+1}} $\\
	$$\int\limits_1^{+\infty} g(x)\dx \sim \int\limits_1^{+\infty}\frac{\cos{7x}\cdot x}{(x^2+1)\sqrt[3]{x^2+1}}$$
	Интеграл справа сходтится $(\frac{5}{3}>1)$. Значит и интеграл слева сходится. 
	\subsection*{б)}
$$\int\limits_0^{+\infty} \frac{1}{\sqrt{e^x-1}}\dx = \int\limits_0^1\frac{1}{\sqrt{e^x-1}}\dx + \int\limits_1^{+\infty}\frac{1}{\sqrt{e^x-1}}\dx$$
$$\lim_{x\to+\infty}\frac{x^2}{\sqrt{e^x-1}} = 0\to \frac{1}{\sqrt{e^x-1}}< \frac{1}{x^2} - \text{сходтися от единицы до бесконечности}$$
$$g(x) = \frac{1}{\sqrt x}, \lim_{x\to\infty}\frac{f(x)}{g(x)} = 1$$
$$g(x)\sim f(x)$$ 
$g(x)$ сходитя от нуля до единицы, значит и $f(x)$
сходится.
\subsection*{c)}
$$\int\limits_1^{+\infty}\ln{(1+\sin{\frac1x})}\dx$$
$$\lim_{x\to\infty}\frac{\ln{(1+\sin{\frac1x})}}{x^{-1}} = 1$$
$$\int\limits_1^{+\infty} \frac1x \text{- расходится, значит, и изначальный интеграл расходится}$$
\section*{d)}
$$\int\limits_0^{+\infty}\frac{\ln(1+x)}{x^a} =\int\limits_0^{1}\frac{\ln(1+x)}{x^a} + \int\limits_1^{+\infty}\frac{\ln(1+x)}{x^a} $$

\end{document}