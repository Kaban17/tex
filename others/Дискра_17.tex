\documentclass[a4paper,12pt]{article}
\usepackage[utf8]{inputenc}
\usepackage[english,russian]{babel}
\usepackage[T2A]{fontenc}
\usepackage{mathtext}
\usepackage{gauss}
\usepackage{graphicx}
\usepackage{amsmath, amsfonts, amssymb}
\newtheorem{theorem}{Теорема}
\usepackage[left=2.50cm, right=2.00cm, top=2.00cm, bottom=2.00cm]{geometry} 
\usepackage{mathdots} 
\usepackage[pdftex]{lscape}
\usepackage{mathtools}
\usepackage{pgfplots}
\pgfplotsset{compat=1.9}
\usepackage{graphicx}%Вставка картинок правильная
\usepackage{tikz}
\usepackage{float}%"Плавающие" картинки
 \usepackage{relsize}
\usepackage{wrapfig}%Обтекание фигур (таблиц, картинок и прочего)
\usepackage{ tipa }
\usepackage{amsmath}
  \usepackage[unicode=true, colorlinks=true, linkcolor=blue, urlcolor=blue]{hyperref}
\linespread{1}
\newcommand{\om}{\overline{o}}
\newcommand{\OB}{\underline{O}}
\newcommand{\eps}{\varepsilon}
\newcommand{\RR}{\mathbb{R}}
\newcommand{\NN}{\mathbb{N}}
\newcommand{\CC}{\mathbb{C}}
\newcommand{\QQ}{\mathbb{Q}}
\newcommand{\ZZ}{\mathbb{Z}}
\newcommand{\dx}{\d{dx}}
\newcommand{\ph}{\varphi}
\newcommand{\F}{\mathbb{F}}
\newcommand{\E}{\mathbb{E}}
\begin{document}
	\section*{17.2}
	\subsection*{а)}
	Проверим  основые свойста для пересечения порядков.\\
	Если элемент порядка 1 сравним с другим элементом и и этот тот же элемент сравним во втором порядке, то и их пересечение порядоков оставляет сравнимость.\\
	Если $(aR_1b\wedge bR_1c) \wedge((aR_2b\wedge bR_2c) )\to (a(R_1\cap R_2)b\wedge b(R_1\cap R_2)c)$  $\to a(R_1\cap R_2)c$ \\
	 Рефлексивность или антирефлексивность очевидна. 
	\section*{17.4}
	\subsection*{а)}
	Так как при изоморфизме предельный переходят в предельные, и при изоморфизме сохраняется порядок, то далекие переходят в далекие.
	\subsection*{б}
	$$A = \ZZ \cup (0,1) \cup (2,3)$$
	Рассмотрим порядок $(A,<)$. В этом порядке есть далекие элементы. В этом порядке есть предельные элементы. Элемент 0.5 подходит под условие задача. Он не предельный  и больше бесконечного количества предельных элементов и меньше бесконечного количества.	
\end{document}
