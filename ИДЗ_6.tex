\documentclass[a4paper,12pt]{article}
\usepackage[utf8]{inputenc}
\usepackage[english,russian]{babel}
\usepackage[T2A]{fontenc}
\usepackage{mathtext}
\usepackage{gauss}
\usepackage{graphicx}
\usepackage{amsmath, amsfonts, amssymb}
\newtheorem{theorem}{Теорема}
\usepackage[left=2.50cm, right=2.00cm, top=2.00cm, bottom=2.00cm]{geometry} 
\usepackage{mathdots} 
\usepackage[pdftex]{lscape}
\usepackage{mathtools}
\usepackage{pgfplots}
\pgfplotsset{compat=1.9}
\usepackage{graphicx}%Вставка картинок правильная
\usepackage{tikz}
\usepackage{float}%"Плавающие" картинки
 \usepackage{relsize}
\usepackage{wrapfig}%Обтекание фигур (таблиц, картинок и прочего)
\usepackage{ tipa }
\usepackage{amsmath}
  \usepackage[unicode=true, colorlinks=true, linkcolor=blue, urlcolor=blue]{hyperref}
\linespread{1}
\newcommand{\om}{\overline{o}}
\newcommand{\OB}{\underline{O}}
\newcommand{\eps}{\varepsilon}
\newcommand{\RR}{\mathbb{R}}
\newcommand{\NN}{\mathbb{N}}
\newcommand{\CC}{\mathbb{C}}
\newcommand{\QQ}{\mathbb{Q}}
\newcommand{\ZZ}{\mathbb{Z}}
\newcommand{\dx}{\d{dx}}
\newcommand{\ph}{\varphi}
\newcommand{\F}{\mathbb{F}}
\newcommand{\E}{\mathbb{E}}
\begin{document}
	\section*{1)}
	Найдем координаты $1-3x+3x^2$ в заданном базисе. Для этого нам надо решить СЛУ:  \\
	$$1-3x+3x^2 = \alpha_1(-2+4x-x^2) + \alpha_2(1-3x+2x^2) + \alpha_3(-2+2x+3x^2)$$
	$$\begin{pmatrix}
		-2 & 1 & -3 & 1 \\
		4 & -3 & 2 & -3 \\
		-1 & 2 & 3 & 3 
	\end{pmatrix} \to \begin{pmatrix}
	1 & 0 & 0 & -2 \\
	0 & 1 & 0 & -1 \\
	0 & 0 & 1 & 1 
	\end{pmatrix}$$
	Теперь найдем $\varphi(1-3x+3x^2)$ \\
	$$\varphi(1-3x+3x^2) = \begin{pmatrix}
		1 & 7 & 3 \\
		-1 & 2 & 5 
	\end{pmatrix}\cdot (-2,-1,1)^T = (-6, 5)^T $$
	Найдем координаты полученного вектора в данном нам базисе. \\
	$$(-6,5) = a_1\cdot(3,2) + a_2\cdot(-2,-1) \to a_1 = 16, a_2 = 27$$
	\section*{2)}
	\subsection*{a)}
	Запишем векторы а в столбцы матрицы и приведем ее к УСВ. \\
	$$\begin{pmatrix}
		-2 & -3 & -3 & 0 & -2 \\
		-2 & 1 & 4 & -1 & -3 \\
		1 & -1 & -3 & -4 & 2 \\
		4 & -1 & -4 & 0 & 1 \\
		1 & 3 & 5 & 4 & 4 
	\end{pmatrix}\to\begin{pmatrix}
	1 & 0 & 0 & 0 & 0 \\
	0 & 1 & 0 & 0 & 0\\
	0 & 0 & 1 & 0 & 0 \\
	0 & 0 & 0 & 1 & 0 \\
	0 & 0 & 0 & 0 & 1 
	\end{pmatrix}$$ $ \text{Следовательно, вектора $a_1-a_5$ ЛНЗ и их 5, они являются базисами в$  \RR^5$. Так какЛО однозначно задается базисными векторами, то ЛО единсвтенное  } $. 
	\subsection*{б)}
	Запишем матрицу линейного отображения.  
	$$\begin{pmatrix}
		36 & -73 & -139 & 6 & 84 \\
		-20 & 43 & 82 & 1 & -49 \\
		-36 & 73 & 139 & -6 & -84
	\end{pmatrix}$$
Составим для этой матрицы ФСР и найдем ЛНЗ столбцы. ФСР будет образовывать базис ядра, а столбцы - образ. 
	$$\begin{pmatrix}
	36 & -73 & -139 & 6 & 84 \\
	-20 & 43 & 82 & 1 & -49 \\
	-36 & 73 & 139 & -6 & -84
\end{pmatrix}\to \begin{pmatrix}
1 & 0 & \frac{9}{88} & \frac{331}{88} & \frac{35}{88} \\
0 & 1 & \frac{43}{22} & \frac{39}{22} & -\frac{21}{22} \\
0 & 0 & 0 & 0 & 0
\end{pmatrix}$$
$A_1 = \begin{pmatrix}
	36 \\
	-20 \\
	-36 
\end{pmatrix} ,A_2 = \begin{pmatrix}
-73 \\
43\\
73 
\end{pmatrix} $ базис в $Im\varphi$ \\
$\begin{pmatrix}
	-\frac{9}{88} \\
	-\frac{43}{22} \\
	1 \\
	0 \\
	0
\end{pmatrix},
\begin{pmatrix}
	\frac{33}{88} \\
	\frac{39}{22} \\
	0 \\
	1 \\
	0
\end{pmatrix} ,
\begin{pmatrix}
	-\frac{35}{88} \\
	-\frac{21}{22} \\
	0 \\
	0 \\
	1
\end{pmatrix}$ базис в $\ker\varphi$
\section*{4}
Найдем базис ядра и дополним его до базиса всего пространсва $\RR^5$ \\
$$\begin{pmatrix}
	62 & 18 & -1 & 44 & 2 \\
	34 & 18 & -4 & 28 & 0 \\
	-51 & 4 & -7 & -27 & -4 \\
	-52 & -10 & -1 & -34 & -2 \\
\end{pmatrix}\to \begin{pmatrix} 
1 & 0 & 0 & \frac{17}{41} & -\frac{2}{41} \\
0 & 1 & 1 & \frac{45}{41} & \frac{14}{41} \\
0 & 0 & 1 & \frac{60}{41} & \frac{46}{41}
\end{pmatrix}$$
$$\text{ФСР:} a_1 = \begin{pmatrix}
	-\frac{17}{41} \\
	-\frac{45}{41} \\
	-\frac{60}{41} \\
	1 \\
	0
\end{pmatrix}, a_2 = \begin{pmatrix}
\frac{2}{41} \\
-\frac{14}{41} \\
\frac{46}{41} \\
0 \\
1
\end{pmatrix}$$
Вектора $e_3,e_4,e_5$ дополняют базис ядра до базиса всего пространтсва.
$$\begin{pmatrix}
	-\frac{17}{41} & \frac{2}{41}  & 0 & 0 & 0 \\
	-\frac{45}{41}& -\frac{14}{41} & 0 & 0 & 0\\
	-\frac{60}{41}& \frac{46}{41} &1 & 0 & 0\\
	1 & 0 &0 & 1 & 0\\
	0& 1 & 0 & 0 & 1
\end{pmatrix}\to \begin{pmatrix}
1& 0 & 0 & 0& 0 \\
0 & 1 & 0 & 0 & 0\\
0 & 0 & 1 & 0 & 0\\
0 & 0 & 0& 1 & 0\\
0 & 0 & 0 & 0 & 1
\end{pmatrix}$$
Вектора  $a_1,a_2,e_3,e_4,e_5$ - искомый базис для пространтсва $\RR^5$
Теперь найдем  $\varphi(e_3),\varphi(e_4),\varphi(e_5)$
И дополним эти вектора до базиса $\RR^4$
$$\varphi(e_3) = A\cdot e_3 = \begin{pmatrix}
	-1 \\
	-4\\
	-7\\
	-1
\end{pmatrix}
\varphi(e_4) = \begin{pmatrix}
	44\\
	28 \\
	-27\\
	-34 
\end{pmatrix}
\varphi(e_5) = \begin{pmatrix}
2\\
0\\
-4 \\
-2 
\end{pmatrix}$$
Вектор $(0, 0, 0, 1)^T $ дополняте эти векторы до базиса всего пространства. \\
$$\begin{pmatrix}
-1 & 44 & 2 & 0  \\
-4 & 28 & 0 & 0 \\
-7 & -27 & -4 & 0\\
-1 & -34 & -2 & 1 
\end{pmatrix}\to \begin{pmatrix}
1 & 0 & 0 & 0 \\
0 & 1 & 0 & 0 \\
0 & 0 & 1 & 0 \\
0 & 0 & 0 & 1\end{pmatrix}$$
Эти вектора и образуют искомый базис.\\
$$C_1 = \begin{pmatrix}
	-1 & 44 & 2 & 0  \\
	-4 & 28 & 0 & 0 \\
	-7 & -27 & -4 & 0\\
	-1 & -34 & -2 & 1 
\end{pmatrix}, C_2 = \begin{pmatrix}
-\frac{17}{41} & \frac{2}{41} & 0 & 0 & 0 \\
-\frac{45}{41} & -\frac{14}{41} & 0 & 0 & 0\\
\frac{60}{41} & \frac{46}{41} & 1 & 0 & 0 \\
1 & 0 & 0 & 1 & 0 \\
0 & 1 & 0 & 0 & 1
\end{pmatrix}$$ 
Диагональный вид матрицы: $$D = \begin{pmatrix}
	\frac{120}{41} & \frac{92}{41} & 1 & 0 & 0 \\
	0 & 0 & 0 & 1 & 0 \\
	0 & 0 & 0 & 0 & 1 \\
	0 & 0 & 0 & 0 & 0
\end{pmatrix}$$
$$A = C_1\cdot D \cdot C_2^{-1}$$
\end{document}