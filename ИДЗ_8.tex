\documentclass[a4paper,12pt]{article}
\usepackage[utf8]{inputenc}
\usepackage[english,russian]{babel}
\usepackage[T2A]{fontenc}
\usepackage{mathtext}
\usepackage{gauss}
\usepackage{graphicx}
\usepackage{amsmath, amsfonts, amssymb}
\newtheorem{theorem}{Теорема}
\usepackage[left=2.50cm, right=2.00cm, top=2.00cm, bottom=2.00cm]{geometry} 
\usepackage{mathdots} 
\usepackage[pdftex]{lscape}
\usepackage{mathtools}
\usepackage{pgfplots}
\pgfplotsset{compat=1.9}
\usepackage{graphicx}%Вставка картинок правильная
\usepackage{tikz}
\usepackage{float}%"Плавающие" картинки
 \usepackage{relsize}
\usepackage{wrapfig}%Обтекание фигур (таблиц, картинок и прочего)
\usepackage{ tipa }
\usepackage{amsmath}
  \usepackage[unicode=true, colorlinks=true, linkcolor=blue, urlcolor=blue]{hyperref}
\linespread{1}
\newcommand{\om}{\overline{o}}
\newcommand{\OB}{\underline{O}}
\newcommand{\eps}{\varepsilon}
\newcommand{\RR}{\mathbb{R}}
\newcommand{\NN}{\mathbb{N}}
\newcommand{\CC}{\mathbb{C}}
\newcommand{\QQ}{\mathbb{Q}}
\newcommand{\ZZ}{\mathbb{Z}}
\newcommand{\dx}{\d{dx}}
\newcommand{\ph}{\varphi}
\newcommand{\F}{\mathbb{F}}
\newcommand{\E}{\mathbb{E}}
\begin{document}
	\section*{1}

	$$\begin{pmatrix}
		-4 & 1 & 0 & 0 \\
		-9& 0 & 1 & 0 \\
		-6& 0 & 0 & 1 \\
		6 & 0 & 0 & 1
	\end{pmatrix} \text { - набор ЛНЗ векторов и следовательо базис } \RR^4$$
	Воспользуемся методом ортогонализации Грама-Шмидта и сразу нормируем базис. 
	$$b_1 = a_1$$
	$$b_2 = a_2 - \frac{(a_2,b_1)}{(b_1, b_1)}\cdot b_1 = \begin{pmatrix}\frac{3\sqrt{17}}{13}\\
	\frac{-12}{\sqrt{17}13}\\
	-\frac{8}{\sqrt{17}13} \\
	\frac{8}{\sqrt{17}13}
	\end{pmatrix}$$
	$$b_3 =a_3 - \frac{(a_3,b_2)}{(b_2, b_2)}\cdot b_2 = 
	\begin{pmatrix}
		0\\
		2\sqrt{\frac{2}{17}}\\
		-\frac{3}{\sqrt{34}} \\
			\frac{3}{\sqrt{34}}
			
	\end{pmatrix} $$
	$$b_4 =a_4 - \frac{(a_4,b_3)}{(b_3, b_3)}\cdot b_3 = 
\begin{pmatrix}
0 \\
0 \\
\frac{1}{\sqrt2}\\
\frac{1}{\sqrt2}\\
\end{pmatrix} $$
\section*{2}
$$\begin{pmatrix}2 & 1 & 1 & -2 \\
	-1 & 4 & 4 & 1 
\end{pmatrix}\to\begin{pmatrix}1 & 0 & 0 & -1 \\
0& 1 &  1&0 
\end{pmatrix}$$
$$\text{ ФСР: }\begin{pmatrix}0 & -1 & 1 & 0 \\
	1& 0 &  0&1 
\end{pmatrix} $$
$$\text{pr}_Uv = \text{ сумма проекций на базисные векторы} = -2b_1 -3b_2 = \begin{pmatrix}-3 \\ 2\\ -2 \\-3 \end{pmatrix}$$
$$\text{ort}_Uv = \begin{pmatrix}-1 \\ 2\\ 2 \\1 \end{pmatrix}$$
$$\rho(V, U) = \sqrt{|ort_Uv|} = \sqrt{10}$$
\section*{4}
$$\det A = 5$$
Найдем $a_1, a_2, a_3: -3+3x-4x^2 = a_1(-4+5x-x^2)+ a_2(16-18x+3x^2) +a_3(8-14x+3x^2)$
Составим и СЛУ и решим его. $a_1 = 72/2, a_2 = 55/8, 29/8$\\
Аналогично и для других векторов. 
$a_1 = 87/2, 59/8, 49/8$\\
$a_1 = -82, -65/4, -31/4$\\
$$\det(a_1\frac{71}{2} + a_2\frac{55}{8} +a_3\frac{29}{8} |a_1\frac{87}{2} + a_2\frac{59}{8} +a_3\frac{49}{8}| -82a_1 + a_2\frac{-65}{4} +a_3\frac{-31}{4} ) = $$
$$\det(a_1\frac{71}{2}| a_2\frac{59}{8} |a_3\frac{-31}{4} ) +\det( a_2\frac{55}{8}| a_1\frac{87}{2}| a_2\frac{-65}{4} ) +  \det( a_3\frac{29}{8} |a_3\frac{49}{8}| -82a_1  ) = -\frac{5\cdot71\cdot59\cdot31}{64} $$
\section*{5}
Вычтем из каждого точки $v_i$ точку $v_0$, чтобы получилось подпространоство и также из точки $v$. И найдем проекцию получившейся точки на получившееся подпространство. \\
$$\text{pr}_{L'}v^{'} = \text{сумма проекция на каждый из векторов. } = \frac{80}{179}\begin{pmatrix}1\\7\\ -8\\ -4\\ 7\\\end{pmatrix} + \frac{85}{115}\begin{pmatrix}-4\\-7\\ 0\\ -1\\ 7\\\end{pmatrix} + \frac{105}{115}\begin{pmatrix}-7\\1\\ 0\\ -7\\ 4\\\end{pmatrix} =  $$
$$=\begin{pmatrix}-\frac{31633}{4117}\\-\frac{5378}{4117}\\ -\frac{640}{179}\\ -\frac{31704}{4117}\\ -\frac{46353}{4117}\\\end{pmatrix}$$
Так как $\text{pr}_{L'}v^{'}$ лежит в $L'$, то ближайшая точка к $v'$ будет $\text{pr}_{L'}v^{'}$ \\
Осталось добавить только вычтенный вектор. \\
$$\begin{pmatrix}-\frac{31633}{4117}\\-\frac{5378}{4117}\\ -\frac{640}{179}\\ -\frac{31704}{4117}\\ -\frac{46353}{4117}\\\end{pmatrix} + \begin{pmatrix}2 \\ 3 \\8 \\-7 \\-6\end{pmatrix}$$
 Расстояние от точки до точки не изменится. 
$$\rho(v', \text{pr}_{L'}v^{'}) = \sqrt{\sum_{i=1}^5 (v'_i, \text{pr}_{L'_i}v^{'})^2}$$
\end{document}