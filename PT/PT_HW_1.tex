\documentclass[a4paper,12pt]{article}
\usepackage[utf8]{inputenc}
\usepackage[english,russian]{babel}
\usepackage[T2A]{fontenc}
\usepackage{mathtext}
\usepackage{gauss}
\usepackage{graphicx}
\usepackage{amsmath, amsfonts, amssymb}
\newtheorem{theorem}{Теорема}
\usepackage[left=2.50cm, right=2.00cm, top=2.00cm, bottom=2.00cm]{geometry} 
\usepackage{mathdots} 
\usepackage[pdftex]{lscape}
\usepackage{mathtools}
\usepackage{pgfplots}
\pgfplotsset{compat=1.9}
\usepackage{graphicx}%Вставка картинок правильная
\usepackage{tikz}
\usepackage{float}%"Плавающие" картинки
 \usepackage{relsize}
\usepackage{wrapfig}%Обтекание фигур (таблиц, картинок и прочего)
\usepackage{ tipa }
\usepackage{amsmath}
  \usepackage[unicode=true, colorlinks=true, linkcolor=blue, urlcolor=blue]{hyperref}
\linespread{1}
\newcommand{\om}{\overline{o}}
\newcommand{\OB}{\underline{O}}
\newcommand{\eps}{\varepsilon}
\newcommand{\RR}{\mathbb{R}}
\newcommand{\NN}{\mathbb{N}}
\newcommand{\CC}{\mathbb{C}}
\newcommand{\QQ}{\mathbb{Q}}
\newcommand{\ZZ}{\mathbb{Z}}
\newcommand{\dx}{\d{dx}}
\newcommand{\ph}{\varphi}
\newcommand{\F}{\mathbb{F}}
\newcommand{\E}{\mathbb{E}}
\begin{document}	
\begin{titlepage}

	
	\begin{center}
		\textsc{\textbf{Теория веротностей, дз 1}}
	\end{center}
	
	\vspace{6em}
	
	
	
	\newbox{\lbox}
	\savebox{\lbox}{\hbox{Пупкин Иван Иванович}}
	\newlength{\maxl}
	\setlength{\maxl}{\wd\lbox}
	\hfill\parbox{11cm}{
		\hspace*{5cm}\hspace*{-5cm}Студент:\hfill\hbox to\maxl{Кондратьев Никита \hfill}\\
		\hspace*{5cm}\hspace*{-5cm}Группа:\hfill\hbox to\maxl{238}\\
		}
	
	
	
\end{titlepage}

	\section*{1}
	\subsection*{a)}
	Всего есть 4 различных значения $N = \{0,1,2,3\}$. То есть Вася может решить либо ноль либо 1 либо 2 либо 3 задачи. \\
	$P(0) = (1-0.1)\cdot(1-0.2)\cdot(1-0.3) = 0.504$ . Вася не решил ни первую, ни вторую, ни третью задачи. \\
	$P(1) = 0.1\cdot(1-0.2)\cdot(1-0.3)+0.2\cdot(1-0.1)\cdot(1-0.3) + 0.3\cdot(1-0.1)\cdot(1-0.2) = 0.398 $ \\
	$P(2) = 0.1\cdot0.2\cdot(1-0.3)+ 0.1\cdot(1-0.2)\cdot0.3+ (1-0.1)\cdot0.2\cdot0.3 =0.092$ \\
	$P(3) = 0.1\cdot0.2\cdot0.3 = 0.006$
	\subsection*{b)}
		$$P(N>1) = P(N=2)+P(N=3) = 0.098$$
		$$\E(N) = 0\cdot0.504+ 1\cdot0.398+2\cdot0.092+3\cdot0.006 = 0.6$$	
		$$\E(N^2) = 0\cdot0.504+ 1^2\cdot0.398+ 4\cdot0.092+9\cdot0.006 =0.82 $$
	\section*{2}
	Если количество баллов случайная величина $\eps$ равновероятно распределенная на отрезке $\left[ 1,n\right] $, то $P(\eps) = \frac1n$ \\
	$$\mathbb{E}(\varepsilon) = \frac{1}{n}\sum^n_{i=1}i = \frac{1}{n}\frac{n(n+1)}{2} = \frac{n+1}{2}$$
	$$\mathbb{E}(\varepsilon^2) = \frac{1}{n}\sum^n_{i=1}i^2 = \frac{1}{n}\frac{n(n+1)(2n+1)}{6} = \frac{(n+1)(2n+1)}{6}$$
	$$\mathbb{E}(\varepsilon) = \frac{1}{n}\sum^n_{i=1}i^3 = \frac{1}{n}\frac{n^2(n+1)^2}{4} = \frac{n(n+1)^2}{4}$$
	\section*{3}
	\href{https://pastebin.com/2Snuhsqt}{Решение задачи } \\
	\textbf{Ответ: а) S = -1.833... г) не отвергам гипотезу	\documentclass[a4paper,12pt]{article}
		\usepackage[utf8]{inputenc}
		\usepackage[english,russian]{babel}
		\usepackage[T2A]{fontenc}
		\usepackage{mathtext}
		\usepackage{gauss}
		\usepackage{graphicx}
		\usepackage{amsmath, amsfonts, amssymb}
		\newtheorem{theorem}{Теорема}
		\usepackage[left=2.50cm, right=2.00cm, top=2.00cm, bottom=2.00cm]{geometry} 
		\usepackage{mathdots} 
		\usepackage[pdftex]{lscape}
		\usepackage{mathtools}
		\usepackage{pgfplots}
		\pgfplotsset{compat=1.9}
		\usepackage{graphicx}%Вставка картинок правильная
		\usepackage{tikz}
		\usepackage{float}%"Плавающие" картинки
		\usepackage{relsize}
		\usepackage{wrapfig}%Обтекание фигур (таблиц, картинок и прочего)
		\usepackage{ tipa }
		\usepackage{amsmath}
		\usepackage[unicode=true, colorlinks=true, linkcolor=blue, urlcolor=blue]{hyperref}
		\linespread{1}
		\newcommand{\om}{\overline{o}}
		\newcommand{\OB}{\underline{O}}
		\newcommand{\eps}{\varepsilon}
		\newcommand{\RR}{\mathbb{R}}
		\newcommand{\NN}{\mathbb{N}}
		\newcommand{\CC}{\mathbb{C}}
		\newcommand{\QQ}{\mathbb{Q}}
		\newcommand{\ZZ}{\mathbb{Z}}
		\newcommand{\dx}{\d{dx}}
		\newcommand{\ph}{\varphi}
		\newcommand{\F}{\mathbb{F}}
		\newcommand{\E}{\mathbb{E}}
		
		\section*{1}
		\textbf{Замкнутым брусокм } (промежуток , координатный промежуток в $\R^n$) будем называть множество $I= \{x\in \R^n | a_i \leq x_i|leqb_i, i=1\dots,n\}$
	}
\end{document}