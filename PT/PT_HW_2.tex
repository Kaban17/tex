\documentclass[a4paper,12pt]{article}
\usepackage[utf8]{inputenc}
\usepackage[english,russian]{babel}
\usepackage[T2A]{fontenc}
\usepackage{mathtext}
\usepackage{gauss}
\usepackage{graphicx}
\usepackage{amsmath, amsfonts, amssymb}
\newtheorem{theorem}{Теорема}
\usepackage[left=2.50cm, right=2.00cm, top=2.00cm, bottom=2.00cm]{geometry} 
\usepackage{mathdots} 
\usepackage[pdftex]{lscape}
\usepackage{mathtools}
\usepackage{pgfplots}
\pgfplotsset{compat=1.9}
\usepackage{graphicx}%Вставка картинок правильная
\usepackage{tikz}
\usepackage{float}%"Плавающие" картинки
 \usepackage{relsize}
\usepackage{wrapfig}%Обтекание фигур (таблиц, картинок и прочего)
\usepackage{ tipa }
\usepackage{amsmath}
  \usepackage[unicode=true, colorlinks=true, linkcolor=blue, urlcolor=blue]{hyperref}
\linespread{1}
\newcommand{\om}{\overline{o}}
\newcommand{\OB}{\underline{O}}
\newcommand{\eps}{\varepsilon}
\newcommand{\RR}{\mathbb{R}}
\newcommand{\NN}{\mathbb{N}}
\newcommand{\CC}{\mathbb{C}}
\newcommand{\QQ}{\mathbb{Q}}
\newcommand{\ZZ}{\mathbb{Z}}
\newcommand{\dx}{\d{dx}}
\newcommand{\ph}{\varphi}
\newcommand{\F}{\mathbb{F}}
\newcommand{\E}{\mathbb{E}}
\begin{document}	
	\begin{titlepage}
		
		
		\begin{center}
			\textsc{\textbf{Теория вероятностей, дз 2}}
		\end{center}
		
		\vspace{6em}
		
		
		
		\newbox{\lbox}
		\savebox{\lbox}{\hbox{Пупкин Иван Иванович}}
		\newlength{\maxl}
		\setlength{\maxl}{\wd\lbox}
		\hfill\parbox{11cm}{
			\hspace*{5cm}\hspace*{-5cm}Студент:\hfill\hbox to\maxl{Кондратьев Никита \hfill}\\
			\hspace*{5cm}\hspace*{-5cm}Группа:\hfill\hbox to\maxl{238}\\
		}
		
		
		
	\end{titlepage}
	\section*{1}
	Рассмотрим случайню величину $X_{T_i} = \begin{cases}
		1, \text{ если } i, i+1, i+2 = THT\\
		0,\text{ иначе }
	\end{cases}$ \\
	И величину $X_{H_i} = \begin{cases}
		1, \text{ если } i, i+1, i+2\dots , i+5= HHHHHH\\
		0,\text{ иначе }
	\end{cases}$\\
	Посчитаем вероятность того что $X_{T_i}=1$. Просто перемножим вероятности $P(X_{T_i}=1 ) = 0.2\cdot0.8\cdot0.2 = 0.032$ . По аналогии  $P(X_{H_i}=1 ) =0.262144 $\\
	$$\mathbb{E}(X_{T_i}) =0.032,\mathbb{E}(X_{T_i}) =0.262144 $$
	Пусть ожидаемый выигрыш равен $U$. Так как мы можем сделать либо +1 либо -1, то $U$ можно разложить так: $$U = \sum\limits_{i=1}^{98} \mathbb{E}(X_{T_i}) - \sum\limits_{i=1}^{95} \mathbb{E}(X_{H_i})  = 98\cdot0.032- 95\cdot 0.262144 =-21.76768$$
	\section*{3}
	\href{https://pastebin.com/AfR4trVC}{ссылка на код}
\end{document}