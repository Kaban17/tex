\documentclass[a4paper,12pt]{article}
\usepackage[utf8]{inputenc}
\usepackage[english,russian]{babel}
\usepackage[T2A]{fontenc}
\usepackage{mathtext}
\usepackage{gauss}
\usepackage{graphicx}
\usepackage{amsmath, amsfonts, amssymb}
\newtheorem{theorem}{Теорема}
\usepackage[left=2.50cm, right=2.00cm, top=2.00cm, bottom=2.00cm]{geometry} 
\usepackage{mathdots} 
\usepackage[pdftex]{lscape}
\usepackage{mathtools}
\usepackage{pgfplots}
\pgfplotsset{compat=1.9}
\usepackage{graphicx}%Вставка картинок правильная
\usepackage{tikz}
\usepackage{float}%"Плавающие" картинки
 \usepackage{relsize}
\usepackage{wrapfig}%Обтекание фигур (таблиц, картинок и прочего)
\usepackage{ tipa }
\usepackage{amsmath}
  \usepackage[unicode=true, colorlinks=true, linkcolor=blue, urlcolor=blue]{hyperref}
\linespread{1}
\newcommand{\om}{\overline{o}}
\newcommand{\OB}{\underline{O}}
\newcommand{\eps}{\varepsilon}
\newcommand{\RR}{\mathbb{R}}
\newcommand{\NN}{\mathbb{N}}
\newcommand{\CC}{\mathbb{C}}
\newcommand{\QQ}{\mathbb{Q}}
\newcommand{\ZZ}{\mathbb{Z}}
\newcommand{\dx}{\d{dx}}
\newcommand{\ph}{\varphi}
\newcommand{\F}{\mathbb{F}}
\newcommand{\E}{\mathbb{E}}
\begin{document}
	\section*{1}
Рассмотрим бесконечные последовательности из $0$, $1$ и $2$, в
которых никакая цифра не встречается два раза подряд. Верно ли, что
мощность множества таких последовательностей имеет мощность континуум? \\
Пусть $s -$ множество всех таких последовательностей. Построим биекцию между $2^{\mathbb{N}} \text{ и } s$ . 00 переходит в 0, 01 переходит в 1, 11 переходит в 2. Следовательно мощность s континуум.
\section*{2}
Рассмотрим множество пар различных действительных чисел, то есть \\
$$\bar D  = \{(x,y) : x\ne y \in \mathbb{R} \}$$  \\Является ли множество $\bar D$ континуальным?
$$ |\mathbb{R}| < |\bar D| < |\mathbb{R}^2| $$  D континуально по теорема Кантора-Бернштейна
\subsection*{3}
Является ли множество всех тотальных функций $\RR\to\RR$ континуальным?
Рассмотрим графики функций. Каждый график функции является подмножеством $\RR^2$. Следовательно мы можем каждую функцию сопоставить с графиком. Следовательно мы выберем все подмножества $\RR^2$, а их $|2^{\RR}|$, а их, в свою очередь, больше, чем $|2^{\NN}|>$ континуум.
\end{document}