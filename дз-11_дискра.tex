\documentclass[a4paper,12pt]{article}
\usepackage[utf8]{inputenc}
\usepackage[english,russian]{babel}
\usepackage[T2A]{fontenc}
\usepackage{mathtext}
\usepackage{gauss}
\usepackage{graphicx}
\usepackage{amsmath, amsfonts, amssymb}
\newtheorem{theorem}{Теорема}
\usepackage[left=2.50cm, right=2.00cm, top=2.00cm, bottom=2.00cm]{geometry} 
\usepackage{mathdots} 
\usepackage[pdftex]{lscape}
\usepackage{mathtools}
\usepackage{pgfplots}
\pgfplotsset{compat=1.9}
\usepackage{graphicx}%Вставка картинок правильная
\usepackage{tikz}
\usepackage{float}%"Плавающие" картинки
 \usepackage{relsize}
\usepackage{wrapfig}%Обтекание фигур (таблиц, картинок и прочего)
\usepackage{ tipa }
\usepackage{amsmath}
  \usepackage[unicode=true, colorlinks=true, linkcolor=blue, urlcolor=blue]{hyperref}
\linespread{1}
\newcommand{\om}{\overline{o}}
\newcommand{\OB}{\underline{O}}
\newcommand{\eps}{\varepsilon}
\newcommand{\RR}{\mathbb{R}}
\newcommand{\NN}{\mathbb{N}}
\newcommand{\CC}{\mathbb{C}}
\newcommand{\QQ}{\mathbb{Q}}
\newcommand{\ZZ}{\mathbb{Z}}
\newcommand{\dx}{\d{dx}}
\newcommand{\ph}{\varphi}
\newcommand{\F}{\mathbb{F}}
\newcommand{\E}{\mathbb{E}}
\begin{document}
	\section*{11}
	\subsection*{Д11.1.}
	\begin{tikzpicture}
		% Вершины графа
		\node[circle, draw] (B) at (0,0) {B};
		\node[circle, draw] (A) at (0,-2) {A};
		\node[circle, draw] (C) at (2,0) {C};
		\node[circle, draw] (D) at (4,0) {D};
		\node[circle, draw] (E) at (4,-2) {E};
		\node[circle, draw] (F) at (2,-2) {F};
		% Рёбра графа
		\draw (B) -- (A);
		\draw (C) -- (D);
		\draw (E) -- (F);
		\draw (E) -- (D);
	\end{tikzpicture} \quad
		\begin{tikzpicture}
		% Вершины графа
		\node[circle, draw] (B) at (0,0) {B};
		\node[circle, draw] (A) at (0,-2) {A};
		\node[circle, draw] (C) at (2,0) {C};
		\node[circle, draw] (D) at (4,0) {D};
		\node[circle, draw] (E) at (4,-2) {E};
		\node[circle, draw] (F) at (2,-2) {F};
		% Рёбра графа
		\draw (B) -- (A);
		\draw (C) -- (B);
		\draw (E) -- (F);
		\draw (E) -- (D);
	\end{tikzpicture} \\
	В лесу на 6 вершинах и 4 ребрах всегда будет 2 компоненты связности. Т.к. в дереве на n вершинах n-1 ребро, то лес на 6 вершинах получится из дерева на 5 ребрах и 6 вершинах удалением одного ребра, которое является мостом.
	\subsection*{Д11.2.}
	Путь является  сочетанием из $C_n^2\cdot2 = n^2-n$. Надо добавить еще пути единичнной длины, их ровно n штук. всего $n^2$ путей \\
 
	\subsection*{Д11.3.}
	Мы знаем, что количество ребер в два раза меньше, чем сумма степенй вершин. Значит, в этом графе 10 ребер. Мы знаем, что в дереве на n вершинах всегда n-1 ребро. Дерево на n ребрах всегда связывает большее количество вершин, чем недерево на n вершинах, т.к в дереве каждое ребро-мост. Значит искомый граф - дерево на 11 вершинах. 
	Пример такого дерева:  \\

	\begin{tikzpicture}
		\node {Корень}
		child {node {$\circ$}}
		child {node {$\circ$}}
		child {node {$\circ$}}
		child {node {$\circ$}}
		child {node {$\circ$}}
		child {node {$\circ$}}
		child {node {$\circ$}}
		child {node {$\circ$}}
		child {node {$\circ$}}
		child {node {$\circ$}};
	
	\end{tikzpicture}
	\subsection*{Д11.4.}
	\subsubsection*{a)}


	\begin{tikzpicture}
		\node {1}
		child {node {2 }
		 child {node {5 }}
		 child {node {6 }}
		 child {node {7 }}
		 child {node {8 }}
		 child {node {9 }}
		 }
		child {node {3 }
		child {node {10 }}
		child {node {11 }}
		child {node {12 }}
		child {node {13 }}	
		}
		child {node {4 }};
	\end{tikzpicture}
		\subsubsection*{b)}
		В дереве на 13 вершинах 12 ребер. Значит сумма степененй вершин 24. Выкиным 2 ребра, сумма степеней, которых 6. Останется сумма степеней-12, 11 вершин. Среди этих вершин должна быть вершина четной степени, иначе будет нечетная сумма. Пусть это четное число = 4, тогда у нас останется 10 вершин с суммой 8. Такого быть не может, т.к граф связный 
\end{document}
