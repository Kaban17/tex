\documentclass[a4paper,12pt]{article}
\usepackage[utf8]{inputenc}
\usepackage[english,russian]{babel}
\usepackage[T2A]{fontenc}
\usepackage{mathtext}
\usepackage{gauss}
\usepackage{graphicx}
\usepackage{amsmath, amsfonts, amssymb}
\newtheorem{theorem}{Теорема}
\usepackage[left=2.50cm, right=2.00cm, top=2.00cm, bottom=2.00cm]{geometry} 
\usepackage{mathdots} 
\usepackage[pdftex]{lscape}
\usepackage{mathtools}
\usepackage{pgfplots}
\pgfplotsset{compat=1.9}
\usepackage{graphicx}%Вставка картинок правильная
\usepackage{tikz}
\usepackage{float}%"Плавающие" картинки
 \usepackage{relsize}
\usepackage{wrapfig}%Обтекание фигур (таблиц, картинок и прочего)
\usepackage{ tipa }
\usepackage{amsmath}
  \usepackage[unicode=true, colorlinks=true, linkcolor=blue, urlcolor=blue]{hyperref}
\linespread{1}
\newcommand{\om}{\overline{o}}
\newcommand{\OB}{\underline{O}}
\newcommand{\eps}{\varepsilon}
\newcommand{\RR}{\mathbb{R}}
\newcommand{\NN}{\mathbb{N}}
\newcommand{\CC}{\mathbb{C}}
\newcommand{\QQ}{\mathbb{Q}}
\newcommand{\ZZ}{\mathbb{Z}}
\newcommand{\dx}{\d{dx}}
\newcommand{\ph}{\varphi}
\newcommand{\F}{\mathbb{F}}
\newcommand{\E}{\mathbb{E}}
\begin{document}
	\section*{1)}
	$$f  = x_1 \vee \dots \vee x_{2024}$$
	$$g =  \neg x_1\dots \vee \neg x_{2024}$$
	$$D(f) = D(g)$$
	$$D(f\vee g) = 0$$
	\section*{2)}
	
	$$D(T_k(x_1,\dots, x_n)) = n$$
	Вот стратегия противника, дающая этот результат. На первые $k-1$ вопрос отвечаем единица. На оставшиеся $n-k+2$ вопросов отвечаем ноль и на последний вопрос отвечаем единицей.
	 \section*{3)}
	 $$D(x_1,\cdots , x_{10} ) = D(x_{11},\cdots , x_{20} ) = 10$$
	 Сложность вычисления обехи дизъюнкций равна 20(нам сначала надо вычислить первую дизъюнкцию, потом вторую). Значить сложность $D(f) = 20$
\end{document}